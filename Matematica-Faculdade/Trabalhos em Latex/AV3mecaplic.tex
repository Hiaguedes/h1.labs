\documentclass[11pt,a4paper]{article}
\title{Trabalho Preparatório}
\author{Hiago Riba Guedes RGU:11620104\\ Lucas Priori RGU:11311093}
\date{Data limite : 24/06/2017}
\usepackage{tikz}
\usepackage[utf8]{inputenc}
\usepackage{gensymb}
\usepackage{pgfplots}
\begin{document}
\maketitle

\textbf{Problema 1}\\
Peso específico do martelo = 0.3 lb.in$^{-3}$\\
Peso específico da mão de madeira=0.9 x 0.036 lbf.in$^{-3}$\\
Dimensão da cabeça $r_d$=0.5 in ; $h_d$=3 in\\
Dimensão da mão $r_{h1}$=0.625 in ; $L_{h1}$=10 in ;
$r_{h2}$=0.3125 in ; $L_{h2}$=2$r_d$ in \\\\
1-Volume e peso dos componentes \\\\
Cabeça:\\
$V_d$=2.049 in$^3$\\
$W_d$=0.615 lbf\\
\\
Mão:\\
$V_{h1}$=12.272 in$^3$\\
$V_{h2}$=0.307 in$^3$\\
$V_h$=12.579 in$^3$\\
$W_{h1}$=0.398 lbf$^3$\\
$W_{h2}$=9.940 x 10$^{-3}$ lbf\\
$W_h$=0.408 lbf\\
\\
2-CG de cada componente no eixo XX\\
\\
Cabeça:\\
$x_{cgh}=L_{hl}+r_d$=10.5 in\\\\
Mão:\\
$x_{cgh}=\frac{0.5L_{hl}V_{hl}+(L_{hl}+0.5L_{h2})V_{h2}}{V_h}$=5.134 in\\\\
3-Achar a localização do composto dos CG's \\\\
$X_{Cg}=\frac{x_{cgd}W_d +x_{cgh}W_h}{W_d + W_h}$=8.361 in \\\\
4-Calcular o momento de inércia no eixo ZZ\\\\
$I_{DDd}=\frac{W_d}{12g}(3r_d^2+h_d^2)=1.294.10^{-3}lbf.sec^2.in$\\
$I_{ZZd}=I_{DDd}+\frac{W_d}{g}x_{cgd}^2=0.177lbf.sec^2.in$\\\\
5-Calcular o momento de inércia da mão no eixo ZZ\\
$l_{h1}=8.683.10^{-3} lbf.sec^2.in$\\
$l_{ZZh1}=0.034 lbf.sec^2.in$\\
$l_{h2}=2.774.10^{-6} lbf.sec^2.in$\\
$l_{ZZh2}=2.841.10^{-3} lbf.sec^2.in$\\
$l_{ZZh}=0.037 lbf.sec^2.in$\\ 
\\
6-Adicionar os momentos dos dois componentes sobre o eixo ZZ pra saber o momento de inércia total do martelo\\
$I_{ZZ}=I_{ZZd}+I_{ZZh}=0.214lbf.sec^2.in$\\
\\
7-Calculor o raio de giro no eixo ZZ\\
\\
$k=\sqrt{\frac{I_{ZZ}g}{W_d + W_h}}=8.992 in$\\
\\
\textbf{Problema 2}\\
\\
$\sum algarismos do RGU=34$\\
Sentido de $\omega_2$ é antihorário\\
Distância entre centros de manivela é de 240 mm \\
AB=501mm

\end{document}