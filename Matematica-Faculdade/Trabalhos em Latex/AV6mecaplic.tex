\documentclass[11pt,a4paper]{article}
\title{Avalia\c{c}\~ao 6 de Mec\^anica Aplicada}
\author{Hiago Riba Guedes RGU:11620104\\ Lucas Priori RGU:11311093}
\date{Data limite : 24/06/2017}
\usepackage{tikz}
\usepackage[utf8]{inputenc}
\usepackage{graphicx}
\usepackage{subfigure}
\graphicspath{{/home/hiago/Desktop/}}
\usepackage{gensymb}
\usepackage{pgfplots}
\begin{document}
\maketitle
\textbf{Problema 1}\\\\
Dados:\\
Subida(cicloidal):25mm em 90 graus\\
Espera superior:25mm em 90 graus\\
Descida(cicloidal):25mm em 90 graus\\
Espera inferior:0mm em 90 graus\\
Came$\omega$:2$\pi$ $\frac{rad}{s}$  = 1 $\frac{rev}{s}$ , sentido horário\\\\
Variação do ângulo ($\theta$) : 15\degree=0.2618 rad\\\\
\begin{tikzpicture}
\draw[very thin,->] (0,0) -- (0,4);%vetores de coordenadas
\draw[very thin,->] (0,0) -- (16,0);%vetores de coordenadas
\draw[dashed] (0,3) -- (12,3);
\draw[dashed] (3,0) -- (3,3);
\draw[dashed] (6,0) -- (6,3);
\draw[dashed] (12,0) -- (12,3);
\draw[dashed] (9,0) -- (9,3);
\draw[line width=0.25mm,red] (0,0) -- (3,0);
\draw[line width=0.25mm,red] (3,3) -- (6,3);
\draw[line width=0.25mm,red] (9,0) -- (12,0);

\node[below left] at (0,0) {\fontsize{10}{10}\selectfont 0};
\node[above left] at (0,3) {\fontsize{10}{10}\selectfont h};
\node[below left] at (1,4.5) {\fontsize{10}{10}\selectfont Movimento};

\node[below left] at (0.3,-1) {\fontsize{10}{10}\selectfont 0};
\node[below left] at (3.3,-1) {\fontsize{10}{10}\selectfont 90};
\node[below left] at (6.3,-1) {\fontsize{10}{10}\selectfont 180};
\node[below left] at (9.3,-1) {\fontsize{10}{10}\selectfont 270};
\node[below left] at (12.3,-1) {\fontsize{10}{10}\selectfont 360};
\node[below left] at (15.9,-1) {\fontsize{7}{7}\selectfont Ângulo do came $\theta$ em graus};

\node[below left] at (0.3,-2) {\fontsize{10}{10}\selectfont 0};
\node[below left] at (3.3,-2) {\fontsize{10}{10}\selectfont 0.25};
\node[below left] at (6.3,-2) {\fontsize{10}{10}\selectfont 0.50};
\node[below left] at (9.3,-2) {\fontsize{10}{10}\selectfont 0.75};
\node[below left] at (12.3,-2) {\fontsize{10}{10}\selectfont 1.0};
\node[below left] at (15.9,-2) {\fontsize{7}{7}\selectfont Tempo (segundos)};

\node at (1.5,1.5){\fontsize{10}{10}\selectfont Subida};
\node at (4.5,3.2){\fontsize{10}{10}\selectfont Espera Superior};
\node at (7.5,1.5){\fontsize{10}{10}\selectfont Descida};
\node at (10.5,1.5){\fontsize{10}{10}\selectfont Espera Inferior};
\end{tikzpicture}
\\\\
Tratando-se de um deslocamento cicloidal temos as seguintes fórmulas para o deslocamento,velocidade,aceleração e pulso:
$$s=h\left[\frac{\theta}{\beta}-\frac{1}{2\pi} sin\left(\frac{2\pi \theta}{\beta}\right)\right]$$
$$v=\frac{h}{\beta}\left[ 1-cos\left(\frac{2\pi \theta}{\beta}   \right)  \right]$$
$$j=4\pi^2 \frac{h}{\beta^3} cos\left(\frac{2\pi \theta}{\beta}   \right)$$
$$a=\frac{2\pi h}{\beta^2} sin\left(2\pi\frac{\theta}{\beta}\right)$$
Com $\theta$ em radianos e $\beta$ indo de 0 até $\theta$ e h=25 mm \\\\
Temos os gráficos:\\
\begin{figure}[h]
\center
\subfigure[ref1][Figura relacionada ao posiciamento]
{\includegraphics[scale=0.3]{s}}
%\caption{Figura relacionada ao posiciamento}
\qquad
\subfigure[ref2][Figura relacionada ao velocidade]
{\includegraphics[scale=0.3]{v}}
\qquad
\subfigure[ref3][Figura relacionada a aceleração]
{\includegraphics[scale=0.3]{a}}
\qquad
\subfigure[ref4][Figura relacionada ao pulso]
{\includegraphics[scale=0.3]{j}}

\end{figure}
\end{document}