\documentclass[11pt,a4paper]{article}
\usepackage[utf8]{inputenc}
\title{Primeiro Trabalho Engenharia de Software - Ferramentas da 4ª geração}
\author{Hiago Riba Guedes RGU:11620104}
\date{Professor:Luís Alexandre}
\usepackage{graphicx}
\usepackage{circuitikz}
\usepackage{pgfplots}  
\usetikzlibrary{shapes,arrows}
\usepackage{subfigure}
\usepackage[margin=20mm]{geometry}
\graphicspath{{/home/hiago/Desktop/}}
\begin{document}
\maketitle
Softwares Escolhidos :Glade e Workbench\\\\
Principais itens a observar:\\\\
Finalidade da ferramenta\\
Características\\
Plataformas\\
Principais vantagens\\
Tempo no mercado\\
Exemplos, caso possível, alguns fabricantes disponibilizam figuras.

\section*{Glade}
Glade é um construtor de interface gráfica para GTK+ que facilita a criação de programas que se integram ao GNOME. Suas vantagens estão no fato de o desenvolvedor conseguir criar uma interface gráfica independente do código. O Glade em si não gera um código , ele gera um arquivo xml e com ele pode-se usar códigos em diversas outras linguagens,cabe salientar aqui que quando desenvolve-se uma interface gráfica o código já precisa estar em mente ou já desenvolvido,então o xml serve apenas como um teste e com o Glade consegue-se deixar aberto diversos projetos ao mesmo tempo . Uma outra vantagem é o fato de seu ambiente de desenvolvimento ser multiplataforma. 

Sua primeira versão data do dia de 18 de abril de 1998
\begin{figure}[!h]
\begin{center}
\includegraphics[scale=0.5]{CNC}
\caption{Exemplo de um painel de controle de CNC feito no Glade}
\end{center}
\end{figure}

\section*{Workbench}
Atualmente na versão 6.3, o Workbench foi desenvolvido para ser um sucessor do DBDesigner4,por isso suas versões começam a partir do 5. Seu lançamento data de 2007.Porém o DBDesigner é uma ferramenta que existe desde 2002.O Workbench é multiplataforma. \\
Sua principal proposta é a de fornecer um ambiente visual limpo e profissional para o desenvolvedor poder administrar, desenvolver e criar seu próprio banco de dados , podendo gerar complexos diagramas DER sem perder o seu entendimento e criar seu esqueleto no seu editor. E ainda ser capaz de migrar seu projeto de um dispositivo para outro graças ao servidor MySql. Além de oferecer melhorias de perfomance e desempenho do que seu antecessor.
\begin{figure}[!h]
\begin{center}
\includegraphics[scale=0.5]{wkbc}
\caption{Exemplo de um diagrama DER de banco de dados gerado no Workbench}
\end{center}
\end{figure}
\end{document}
