\documentclass[11pt,a4paper]{article}
\usepackage[utf8]{inputenc}
\title{Iniciação Científica}
\author{Hiago Riba Guedes}
\date{Data limite : Agosto(?)/2018}
\usepackage{tikz}
\usepackage{gensymb}
\usepackage{pgfplots}
\begin{document}
\maketitle
\textbf{O tempo nunca espera.Ele leva todos igualmente para o mesmo fim.\\Você,que deseja proteger o futuro,por mais limitado que seja...\\\\ À você será dado um ano;\\Siga em frente sem hesitar , com o coração a te guiar.  }\\\\
\begin{center}
Cronograma:
\end{center}
Primeiro Trimestre: Estudo sobre a Teoria de Filtros de Kalman\\
Segundo Trimestre:Estudo do problema de localização e estimação de robôs em ambientes internos \\
Terceiro Trimestre: Aplicação do Filtro de Kalman para o problema de estimação e localização de robôs em ambientes internos\\
Quarto Trimestre: Redação de uma Monografia e possivelmente uma montagem prática do problema\\\\
\begin{center}
\textbf{Idéias de implementação}
\end{center}
Implementar uma interface gráfica autônoma com mapas pseudo-randômicos com o tema do Tartarus de Persona 3 e objetivos pseudo-randômicos e que ele seja capaz de gerar gráficos de resposta do sistema. \\
Programa cheio de comentários e que com ele fique mais fácil explicar o que foi dito na monografia.\\
Ter tempo para poder fazer uma aplicação em um robô real ,para poder dizer as principais diferenças entre o programa e a realidade \\
\section{Introdução}
Em  1960  Rudolph  Emil  Kalman  publicou  um  famoso  artigo  descrevendo  um processo recursivo para solucionar problemas lineares relacionados à filtragem de dados discretos. Sua pesquisa proporcionou contribuições relevantes ajudando a estabelecer basesteóricas sólidas em várias áreas da engenharia de sistemas. Em 1960-1961  Kalman  desenvolveu,  com  colaboração de  Richard  S.  Bucy,  a  versão  em tempo contínuo do filtro de Kalman, que se tornou conhecida como o filtro de Kalman-Bucy. Com o avanço computacional, o filtro de Kalman e suas extensões a problemas não lineares representam o produto mais largamente utilizado dentro da moderna teoria de controle. Filtro de Kalman \\
O  filtro  de  Kalman  é  um  conjunto  de  equações  matemáticas que  constitui  um  processo  recursivo  eficiente  de  estimação, uma  vez  que  o  erro  quadrático  é  minimizado.   Através   da   observação   da   variável   denominada   “variável   de   observação”  outra  variável,  não  observável,  denominada  “variável  de  estado”  pode  ser  estimada  eficientemente.  Podem  ser  estimados  os  estados  passados,  o  estado presente e mesmo previstos os estados futuros.\\ 
O  filtro  de  Kalman  é  um  procedimento  aplicável  quando  os  modelos  estão  escritos  sob  a  forma  espaço-estado.  Além  disso,  o  filtro  de  Kalman  permite  a  estimação  dos  parâmetros  desconhecidos  do  modelo  através  da  maximização da verossimilhança via decomposição do erro de previsão
\section{Sobre o Processing}
\end{document}