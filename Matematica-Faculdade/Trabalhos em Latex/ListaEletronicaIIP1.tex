\documentclass[11pt,a4paper]{article}
\usepackage[utf8]{inputenc}
\title{Lista Exercícios Eletrônica 2}
\usepackage{graphicx}
\usepackage{circuitikz}
\usepackage{pgfplots}  
\usetikzlibrary{shapes,arrows}
\usepackage{subfigure}
\usepackage[margin=20mm]{geometry}
\graphicspath{{/home/hiago/Desktop/Eletronica_2/images/}}
\begin{document}
\maketitle
\textbf{1- }
Fazendo da maneira correta sem aplicar fórmula pronta.\\
\\
$v_+=v_-=v_b$ pelo curto virtual
$$v_a-v_-=R_1\times i_1$$
$$v_--v_0=R_2\times i_2$$

Pela impedância infinita na entrada, $i_1$ é igual a $i_2$ 
Aplicando as relações encontradas temos:
$$v_a-v_b=R_1\times i$$
$$v_b -v_0=R_2\times i$$

Como as correntes são iguais para as duas referências podemos igualar-las.\\
$$\frac{v_a-v_b}{R_1}=\frac{v_b-v_o}{R_2}$$
Explicitando $v_b$ na equação temos:\\
$$\frac{v_b}{R_1}+\frac{v_b}{R_2}=\frac{v_a}{R_1}+\frac{v_o}{R_2}$$
$$v_b \left(\frac{1}{R_1} + \frac{1}{R_2}    \right)-\frac{v_a}{R_1}=\frac{v_0}{R_2}$$
$$v_b\left( \frac{R_2}{R_1}+1 \right)-v_a\left( \frac{R_2}{R_1} \right)=v_o$$
\\\\
\textbf{b)}
Jogando então os valores:\\
$R_1=1 K\Omega$\\
$R_2=100K\Omega$\\
$v_a=10mV$\\
$v_b=20mV$\\
$$20\times 10^{-3} \left( \frac{100}{1} \right)-10 \times 10^{-3}\left( \frac{100}{1}  \right)=1 V$$

Com efeito.\\
Simulando o circuito no LT SPice\\
\begin{figure}[!h]
\begin{center}
\includegraphics[scale=0.3]{ex1}
\end{center}
\end{figure}
\\
\textbf{2) }\\\\
Para fazer esse exercício iremos dividir-lo em três circuitos.\\
\begin{figure}[!h]
\begin{center}
\includegraphics[scale=0.7]{somadorex2}
\caption{Circuito Somador}
\end{center}
\end{figure}
\\
\begin{figure}[!h]
\begin{center}
\includegraphics[scale=0.7]{slaoqex2}
\caption{Circuito sla oq}
\end{center}
\end{figure}
\\
\begin{figure}[!h]
\begin{center}
\includegraphics[scale=0.7]{amplificadorex2}
\caption{CIrcuito Amplificador}
\end{center}
\end{figure}
\\
Onde a saída dos circuitos das figuras 1 e 2 são as entradas do circuito 3.\\
\subsubsection*{Circuito 1}
Resolvendo então o circuito 1 temos:\\
As correntes das três linhas que levam ao nó de $v_-$ se somam e o nó $v_-$ tem a mesma tensão de $v_+$ pelo terra virtual que é 0(zero) Logo:\\
$$\frac{2}{6000}=i_1=333.333 \mu A$$
$$\frac{-6}{6000}=i_2=-1 mA$$
$$\frac{6}{6000}=i_3=1mA $$
Fazendo a somatória das correntes encontramos que a corrente que entra no nó $v_-$ é $333.333 \mu A$\\\\
Aplicando a lei dos nós :\\
$$0-v_{o1}=24\times 10^3 \times 333.333\times 10^{-6} $$
$$v_{o1}=-8 V$$
No simulador:\\
\begin{figure}[!h]
\begin{center}
\includegraphics[scale=0.3]{circuito1ex3}
\caption{Resultado somador}
\end{center}
\end{figure}
\\
\subsubsection*{Circuito 2}
No circuito 2,temos que $v_+$ é igual a $v_{ref}$
$$v_0-v_{ref}=4\times 10^3 \times i$$
$$v_{ref}-0=2\times 10^3 \times i$$
Só que $v_{ref}=-3 V$ e o i é o mesmo.\\
$$\frac{v_o}{v_{ref}}-1=2$$
$$v_o=-9$$
No simulador:\\
\begin{figure}[!h]
\begin{center}
\includegraphics[scale=0.3]{circuito2ex3}
\caption{Resultado comparador}
\end{center}
\end{figure}
\\

\subsubsection*{Circuito 3}
Com os dois circuitos calculados temos que a entrada no ramo de cima é -8 e a entrada no ramo de baixo é -9\\\\

$$-8-(-9)=12\times10^3 \times i$$
$$-9-E_o=6\times10^3 \times i$$
$$i=83.33 \mu A$$
$$E_o=-9.5 V$$\\
\begin{figure}[!h]
\begin{center}
\includegraphics[scale=0.3]{circuito3ex3}
\caption{Resultado Total}
\end{center}
\end{figure}
\\
\newpage
\textbf{3) }
Inicialmente vamos fazer o paralelo entre os dois resistores na retroalimentação. Para podermos achar a tensão entre seus nós . Após isso nos preocuparemos com as correntes.\\

$$v_+=v_-=0$$
$$3-0=1000\times i\rightarrow i=3mA$$
$$0-v_o=\frac{1}{\frac{1}{6\times 10^3}+\frac{1}{3\times 10^3}}\times i\rightarrow v_o=-2\times 10^3 \times 3\times 10^{-3}$$
$$v_o=-6V$$
Para encontrar a corrente no resistor de 6K fazemos:
$$-6-0=6\times 10^3 \times i_r$$
$$i_r=-1mA$$
\begin{figure}[!h]
\begin{center}
\includegraphics[scale=0.3]{ex4}
\caption{Resultado}
\end{center}
\end{figure}
\\
\newpage
\textbf{4) }
Fazendo lei dos nós..\\
$$11-V_1=2K \times i_1$$
$$V_1-0=4K \times i_2$$
$$V_1-V_-=6K \times i_3$$
Sendo que :
$$i_1=i_2+i_3$$
$$v_-=v_+=0$$
Aplicando as relações:
$$11-V_1=2K \times (i_2 +i_3)$$
$$V_1=4K \times i_2$$
$$V_1=6K\times i_3$$
Dividindo a segunda equação pela terceira
$$1=\frac{2}{3}\times \frac{i_2}{i_3}\rightarrow \frac{3}{2}=\frac{i_2}{i_3}\rightarrow 1.5i_3=i_2$$
Aplicando essa relação com a terceira equação na primeira 
$$11-6K\times i_3=2K\times(2.5i_3)$$
$$11=11Ki_3$$
$$i_3=1mA$$
Sendo que essa corrente $i_3$ é a mesma corrente que passa no resistor de 12K.\\
$$0-V_0=12K \times 1m\rightarrow V_0=-12 V$$
\begin{figure}[!h]
\begin{center}
\includegraphics[scale=0.3]{correnteex5}
\caption{corrente inversa ao resistor: -i3}
\end{center}
\end{figure}
\\
\begin{figure}[!h]
\begin{center}
\includegraphics[scale=0.3]{resultadoex5}
\caption{Resultado}
\end{center}
\end{figure}
\\
\newpage
\textbf{5) }
Precisamos fazer a lei dos nós e depois aplicamos os limites para a corrente.\\
Temos um circuito comparador, logo $V_{ref}=6V$
$$V_0-V_{ref}=2K\times i$$
$$V_{ref}-0=2K\times i\rightarrow i=3mA$$
Aplicando essa corrente na primeira equação
$$V_0=6+2K\times 3m\rightarrow V_0=12 V$$
Para $V_0$=12 V,temos duas malhas em paralelo ,uma com um resistor de 6K e outra com um resistor que será projetado.\\
Obrigatoriamente o resistor de 6K consumirá
$$\frac{12}{6\times 10^3}=2mA$$
O projeto pede para que i esteja entre 2mA e 8mA,mas o resistor de 6K já consome 2mA então para esse caso ,RL deverá ser $\infty$.\\
Para o caso de 8mA ,o resistor RL consumirá os 6mA restantes,logo.
$$12=6m\times RL$$
$$RL=2000 \Omega$$
\begin{figure}[!h]
\begin{center}
\includegraphics[scale=0.3]{rl5}
\caption{Resultado}
\end{center}
\end{figure}
\\
\newpage
\textbf{6) }
Esse exercício não é muito diferente do exercício 4 , porém nesse caso temos um seguidor de tensão. Que não adicionará em muito ao circuito,ele casará as duas impedâncias.Mas não modificará em nada sua estrutura.\\
Ao lado esquerdo do seguidor de tensão temos um divisor de tensão.\\

$$6-V_{ref}=6K\times i$$
$$V_{ref}-0=2K\times i$$
$$\frac{6}{V_{ref}}-1=3\rightarrow V_{ref}=1.5 V$$
Tensão de referência essa que será a tensão na saída do seguidor.\\
$$1.5-0=1K \times i$$
$$0-V_o=8K\times i$$
$$V_0=-8K\times 1.5 = -12V$$
\begin{figure}[!h]
\begin{center}
\includegraphics[scale=0.3]{ex6}
\caption{Resultado}
\end{center}
\end{figure}
\\
\newpage
\textbf{7) }
Para essa questão deveremos saber que a corrente em um capacitor tem a seguinte relação.
$$i=C\frac{dV}{dt}$$
Por conta dessa relação temos a corrente que passa no resistor.
$$ i_1=C\frac{dV_0}{dt} $$
Para o resistor temos
$$0-V_s=R_f\times i\rightarrow i=-\frac{V_s}{R_f}$$
Igualando:
$$-\frac{V_s}{R_fC}=\frac{dV_0}{dt}$$
O que nos dá:
$$V_s=-R_fC \frac{dV_0}{dt}$$
Que é a fórmula de um circuito derivador.\\\\
Aplicando os valores dados 
$$V_s=-2\times 10^6 \times 0.01\times 10^{-6} \frac{dV_0}{dt}$$
$$V_s=-0.02\frac{dV_0}{dt}$$
-0.02?
\begin{figure}[!h]
\begin{center}
\includegraphics[scale=0.3]{ex7}
\caption{Resultado}
\end{center}
\end{figure}
\\
\textbf{8) }\\\\
Equações da parte de fora
$$v_i-v_1=10K\times i_1$$
$$v_1-v_o=100K\times i_2$$
$$v_1-v_-=10K\times i_3$$
$$0-V_0=100Ki_3$$
$$i_1=i_2+i_3$$
Onde $$v_-=0$$
Aplicando as relações achamos que:
$$V_i=10K(i_1+i_3)$$
$$V_1=100K(i_2-i_3)$$
Igualando $V_1$
$$100K(i_2-i_3)=10Ki_3$$
$$\frac{i_2}{i_3}-1=0.1$$
$$i_2=1.1i_3$$
Aplicando na relação entre as correntes
$$i_1=2.1i_3$$
Queremos calcular qual o ganho do circuito , isso é $\frac{V_0}{V_i}$
$$\frac{V_0}{V_i}=\frac{-100Ki_3}{10K(3.1i_3)}=-3.225 $$
Isso é, se apicarmos 1 V em sua entrada encontraremos -3.225 V na saída.\\
Simulador.\\
\begin{figure}[!h]
\begin{center}
\includegraphics[scale=0.3]{ex8}
\caption{Resultado}
\end{center}
\end{figure}
\\
\textbf{9) }
\\
Pediu-se que fosse feito um circuito somador inversor. Porém como sabemos ,suas relações se baseiam na escolha dos resistores.\\
E sua fórmula geral pode ser dado na seguinte forma
$$V_o=-\frac{R_f}{R_1}V_1 -\frac{R_f}{V_2}V_2 -\frac{R_f}{R_3}V_3$$
E o exercício pede para que 
$$\frac{R_f}{R_1}=4$$
$$\frac{R_f}{R_2}=1$$
$$\frac{R_f}{R_3}=0.1$$
Sabemos que para fins de projeto os valores dos resistores não podem ser muito baixos, na ordem de poucos ohms, pois senão geraria uma corrente muito alta,aumentando desnecessariamente a sua potência. Nem muito alta,pois ai a impedância infinita de um AO não valeria mais e suas operações estariam erradas.\\
Logo pode-se dizer que $R_f$ vale 30 K$\Omega$.
Feito isso acho os valores de todos os outros resistores ,que são os seguintes
$R_1=7.5 K\Omega$
$R_2= 30 K\Omega$
$R_3= 300 K\Omega$
\\\\
O exercício pede para que façamos o gráfico para quando:
$$V_1=2sen(wt)$$
$$V_2=5$$
$$V_3=-100$$
Substituindo esses valores a saída ficará com a seguinte equação\\
$$V_o=-(8sen(wt)+5-10)$$
$$V_o=5-8sen(wt)$$
Isso é o inverso de um gráfico seno,porém multiplicado 8 vezes e subindo 5 unidades. O que é representado no simulador.Com amplitude indo de -3V a 13 V
\begin{figure}[!h]
\begin{center}
\includegraphics[scale=0.3]{ex9}
\caption{Resultado}
\end{center}
\end{figure}
\\
\newpage
\textbf{10) }
\\
Para o primeiro circuito amplificador temos:
$$V_1-0=R_1i$$
$$0-V_{01}=R_2i$$
$$V_{01}=-\frac{R_2}{R_1}V_1$$
Que será aplicado na entrada do segundo circuito que é um circuito somador.\\
Onde se calculado as correntes de entrada temos:
$$i =\frac{V_2}{R_4}-\frac{R_2}{R_1R_3}V_1$$
..
$$0-V_0=R_5\times i$$
$$V_0=-\frac{R_5V_2}{R_4}+\frac{R_2R_5}{R_1R_3}V_1$$
A questão pede para dimensionarmos 
$$V_0=20V_1 -0.2V_2$$
Isso é 
$$\frac{R_5}{R_4}=0.2$$
$$\frac{R_2R_5}{R_1R_3}=20$$
Ai a partir daqui começamos a dimensionar certos ganhos.Supondo que o ganho do primeiro amplficador seja de 10.Isso é $R_1=1k\Omega$ e $R_2=10k\Omega$ . E com isso dizemos também que $R_5$ é de $20K\Omega$. Ficamos então com $R_4=100 K\Omega$ e $R_3=10 K\Omega$ .Simulando esse circuito no LTSpice,como sempre.
\begin{figure}[!h]
\begin{center}
\includegraphics[scale=0.3]{ex10}
\caption{Resultado}
\end{center}
\end{figure}
\\
\section*{Questões Intermediárias}
\textbf{11) }
\\
Eu nao sei se está escrito que el equer que prove que $IL=\frac{V_{in}}{R}$ ou I1, outra coisa qualquer mas vou fazendo.E o que seria esse R.
$$V_{in}-V_C=R_1I_1$$
$$V_C-0=RlIl$$
$$V_C-V_A=(R_2+R_4)I_4$$
$$V_A=R_3I_3$$
Porém da forma como o desenho está representado $i_3=i_4$
$$i_1=i_l+i_4$$
$$V_C=(R_2+R_3+R_4)I_4$$
$$(R_2+R_3+R_4)I_4=RlIl$$
$$I_4=IL\frac{RL}{(R_2+R_3+R_4)}$$
Logo:
$$I_1=IL\frac{(RL+R_2+R_3+R_4)}{(R_2+R_3+R_4)}$$
Aplicando tudo na primeira equação
$$V_{in}-RLIL=R_1IL\frac{(RL+R_2+R_3+R_4)}{(R_2+R_3+R_4)}$$
$$V_{in}=IL\left(RL+R_1\frac{(RL+R_2+R_3+R_4)}{(R_2+R_3+R_4)}\right)$$
E é isso...\\
Mas tenho a impressão que o circuito ta desenhado errado entao nao sei..\\\\
\textbf{12) }
\\
Pela impedância infinita na entrada da porta inversora. \\
$$i_2=0.5 mA$$
Logo a tensão nesse resistor é de 
$$0.5m\times 7.5K=3.75V$$
Mas ...$v_-$ é 0 ,pelo terra virtual com $v_+$ o que faz com que o ponto oposto seja de -3.75 V.\\
Calculando $i_1$ então
$$0-(-3.75)=2.5Ki_1\rightarrow i_1=1.5mA$$
$i_3$ acaba sendo a soma das duas correntes,o que dá 2mA
$$-3.75-V_b=5K\times 2m\rightarrow V_b=-13.75 V$$
\begin{figure}[!h]
\begin{center}
\includegraphics[scale=0.3]{ex11}
\caption{Resultado da tensão}
\end{center}
\end{figure}
\\
\newpage
\textbf{13) }
\\
$$V_i-V_+=R_2i$$
$$V_+-0=R_3i$$
$$V_i=\left(  1+\frac{R_2}{R_3}  \right)V_+$$
$$V_+=\frac{R_3}{R_2+R_3}V_i$$
Pelo curto virtual temos que $v_-=v_+$
$$0-v_+=R_1i$$
Onde podemos tirar que 
$$i=-\frac{v_+}{R_1}$$
$$v_+-v_o=R_fi$$
$$v_+-R_fi=v_o$$
$$v_+-R_f\frac{-v_+}{R_1}=v_o$$
$$v_+\left(1-\frac{R_f}{R_1}\right)=v_o$$
$$\frac{R_3}{R_2+R_3}V_i\left(1+\frac{R_f}{R_1}\right)=v_o$$
Substituindo os valores
$$\frac{20K}{30K}V_i\left( 1+\frac{70K}{20K}\right)=9.6$$
$$V_i=\frac{9.6}{3}=3.2V$$
\begin{figure}[!h]
\begin{center}
\includegraphics[scale=0.3]{ex12}
\caption{Resultado da tensão}
\end{center}
\end{figure}
\\
\textbf{14) }
\\
a)Chave aberta - Resistor $R_3$ integrado ao circuito\\
$$V_i-v_+=R_3i_3$$
$$V_i-v_+=R_2i_2$$
$$v_+-v_0=R_fi_3$$
Porém $i_3$ tende a zero,então $v_+=V_{in}$
$$V_i-V_i=R_2i_2$$
$$i_2=0$$
Se $i_2$=0;$V_o$=$V_i$ V\\
Em nenhum momento há a queda de tensão\\\\
b)Chave fechada - $v_-=0=v_+$
Amplificador inversor $V_0=-\frac{R_f}{R_2}V_i$
\\\\
\textbf{15) }
$$\frac{R_b}{R_a+R_b}v_3=v_+$$
Relações da malha de cima\\
$$v_1-v_+=R_1i\rightarrow i=\frac{v_1-v_+}{R_1}$$
$$v_+-v_2=R_2i$$
$$v_+-v_2=R_2\frac{v_1-v_+}{R_1}$$
$$v_+=\frac{R_2}{R_1}v_+=v_2+\frac{R_2}{R_1}v_1$$
$$v_+\left( 1+\frac{R_2}{R_1}  \right)=v_2+\frac{R_2}{R_1}v_1$$
$$v_2=\frac{R_b}{R_a+R_b}\left(\frac{R_2}{R_1}+1\right)v_3 -\frac{R_2}{R_1}v_1$$
Pede-se para que :
$$v_2=\frac{v_3}{3}-2v_1$$
Fazendo a associação :
$$R_2=2R_1$$
Fazendo $R_2=20K\Omega$;$R_1=10K\Omega$..\\
Então.. 
$$1+\frac{R_2}{R_1}=3$$
$$\frac{R_b}{R_a+R_b}=\frac{1}{9}$$
$$8R_b=R_a$$
Logo podemos fazer : $R_b=10K\Omega$;$R_a=80K\Omega$
\\\\
\textbf{b) }
Se $R_a$ entrar em curto a equação fica da seguinte forma
$$v_2=\left(\frac{R_2}{R_1}+1\right)v_3 -\frac{R_2}{R_1}v_1$$
Para $v_1=0$\\
$$v_2=\left(\frac{R_2}{R_1}+1\right)v_3 $$
$$v_2=6V$$
\begin{figure}[!h]
\begin{center}
\includegraphics[scale=0.3]{ex15}
\caption{a)}
\end{center}
\end{figure}
\\
\begin{figure}[!h]
\begin{center}
\includegraphics[scale=0.3]{ex15b}
\caption{b)}
\end{center}
\end{figure}
\\
\newpage
\textbf{16) }
A esquerda temos dois amplificadores inversores. Onde o circuto de cima chamarei de 1 e o de baixo de 2.\\
A tensão em seus nós de realimentação são os mesmos da entrada,pelo curto virtual. Logo a tensão do nó acima do resistor de 1K é de 2V. A tensão abaixo é de 4 V. Com isso conseguimos achar a corrente $i_2$ que é de $4-2=1K\times i \rightarrow 2mA$. Como não passa corrente na entrada das entradas de um AmpOp, é certo afirmar que $i_1=i_2=i_3=2mA$.\\
Agora podemos calcular as tensões $V_{02}$ e $V_{01}$.
$$2-V_{02}=4K\times 2m \rightarrow V_{02}=-6 V$$  
$$ V_{01}-4=4K\times 2m \rightarrow V_{01}=12 V$$

Onde $V_{01}$ é a entrada da malha onde encontra-se a malha não inversora. Pois ele se encontra antes de um divisor de tensão de resistores iguais. Logo a porta não inversora recebe uma tensão de $\frac{V_{01}}{2}=6 V$ . Logo as equações para esse circuito .\\
$${V_02}-v_+=4K\times i$$
$$v_+-{V_o}=4K\times i$$
Substituindo os valores encontrados na primeira equação.\\
$$-6-6=4K\times i \rightarrow i=-3mA$$
Substituindo esse valor de corrente na segunda equação ,acharemos $V_0$\\
$$v_+-4K\times i={V_o}$$
$$V_o=18 V$$
\begin{figure}[!h]
\begin{center}
\includegraphics[scale=0.3]{ex16}
\caption{Resultado}
\end{center}
\end{figure}
\\
\newpage
\textbf{17) }O raciocício dessa questão é a mesma da 10.\\\\
\textbf{18) }\\
É dado que $V_i$ é contínua.\\
Pela equação de corrente de um capacitor 
$$i=C\frac{dV}{dt}$$
Chegamos que em corrente contínua a corrente que flui em um capacitor é de 0 A.\\
Não havendo queda de tensão entre seus terminais.\\
Chamarei o nó que liga os resistores ao capacitor $C_1$ de $V_c$ e o nó da porta não inversora de $v_+$\\\\
A saída $v_0$ têm como saída o valor de $v_+$ pelo curto virtual somada com a tensão de $V_c$. \\
$$V_0=V_+ + V_c$$ .\\
Agora aplicando a lei dos nós no sistema achamos as seguintes equações.\\
$$V_i-V_c=R_1\times i$$
$$V_c-V_+=R_2\times i$$
Explicitando a corrente na primeira equação temos.\\
$$i=\frac{V_1-V_c}{R_1}$$
Usando o valor de i na segunda equação encontramos 
$$V_i=\left( \frac{R_1}{R_2}+1  \right)V_c - \frac{R_1}{R_2}V_+$$
A equação do ganho então fica:
$$\frac{V_+ + V_c}{\left(\frac{R_1}{R_2}+1  \right)V_c - \frac{R_1}{R_2}V_+} $$
Agora teria que substituir os valores de $V_+$ e $V_c$ mas nao sei como
\\\\
\textbf{19) }
Esse exercício parece complexo a primeira vista, mas se retirarmos os seguidores de tensão e os resistores de 56R ,nós ficaremos com dois circuitos, um amplificador não inversor e outro amplificador inversor.O primeiro circuito é importante para manter a tensão de cálculo para tensões mais altas,mas para efeito de cálculo o resultado é o mesmo.\\\\
Para o primeiro amplificador temos que $V_-=V_i$
Suas equações são..
$$V_a-0=10K\times i$$
$$-V_a+V_o=90K \times i$$
$$\frac{V_0}{V_a}-1=9$$
$$V_a=10V_i$$
Isso é , a tensão na entrada da malha superior do segundo amplificador operacional é 10 vezes o valor da entrada.O que também é um dos terminais do resistor de $200 \Omega$\\
Para o segundo amplificador inversor temos as seguintes equaçoes\\
$$10V_i-0=20K\times i$$
$$0-v_b=20K \times i$$
$$-\frac{10V_i}{v_b}=1$$
$$v_b=-10V_i$$
Loga a diferença de tensão no resistor é de 
$$10V_i-(-10V_i)=20V_i$$
\begin{figure}[!h]
\begin{center}
\includegraphics[scale=0.3]{ex19}
\caption{Resultado}
\end{center}
\end{figure}
\\
\begin{figure}[!h]
\begin{center}
\includegraphics[scale=0.3]{ex19b}
\caption{Os dois circuitos quando operados no limite teórico do amplificador}
\end{center}
\end{figure}
\\
\textbf{20) }
\\
Assumindo que há um erro nessa questão, pois não há um ponto marcando quem é $V_{02}$.Temos as seguintes equações\\
$$V_{o1}-5=100K\times i_5$$
$$5-3=40K\times i_2 \rightarrow i_2=50\mu A$$
$$5-V_{o2}=20K\times i_4$$
$$V_{o2}-3=20K\times i_3$$
$$3=10K\times i_1\rightarrow i_1=300\mu A$$
$$i_5=i_4+i_2$$
$$i_3+i_2=i_1 \rightarrow i_3=300\mu -50\mu =250\mu A$$
O que já da para achar $V_o2$
$$V_{o2}=20K\times 250\mu +3=8 V$$
$$5-8=20K\times i_4 \rightarrow i_4=-150\mu A$$
$$i_5=-150\mu + 50\mu=-100 \mu A$$
Logo:
$$V_{o1}=5+100K\times -100\mu=-5V$$
\begin{figure}[!h]
\begin{center}
\includegraphics[scale=0.3]{20}
\caption{Resultado}
\end{center}
\end{figure}
\\
\newpage
\textbf{21) }
\textbf{a) }
Para esse comparador ,temos que a entrada $V_in$ está conectado a porta inversora do AmpOp.E o nível de comparação é a porta não inversora.Que está conectado a terra. Logo quando $V_{in}$ (que é senoidal), ultrapassar positivamente o valor de 0V. O amplificadorsatura negativamente. Quando $V_{in}$ ultrapasar 0V negativamente, o comparador satura positivamente.
$$v_o=A(-V_{in})$$
\begin{figure}[!h]
\begin{center}
\includegraphics[scale=0.3]{21a}
\caption{Resultado}
\end{center}
\end{figure}
\\
\newpage
\textbf{b) }
Para esse comparador a saída e a entrada estão em mesma fase , porque $V_{in}$ está na entrada não inversora.E seu valor de comparação é -8 V.Então sempre que $V_{in}$ for menor que -8,o valor encontrado na saída irá saturar negativamente.Se for maior saturará postivamente 
$$v_o=A(v_+ +8)$$
\begin{figure}[!h]
\begin{center}
\includegraphics[scale=0.3]{21b}
\caption{Resultado}
\end{center}
\end{figure}
\\
\newpage
\textbf{c) }
Essa é o oposto da letra b) o valor de $V_{in}$ encontra-se agora na entrada não inversora e o valor de comparação agora é -6V.
$$v_o=A(-6-V_{in})$$
\begin{figure}[!h]
\begin{center}
\includegraphics[scale=0.3]{21c}
\caption{Resultado}
\end{center}
\end{figure}
\\
\newpage
\textbf{d) }
$$v_o=A(4 -V_{in})$$
\begin{figure}[!h]
\begin{center}
\includegraphics[scale=0.3]{21d}
\caption{Resultado}
\end{center}
\end{figure}
\\
\newpage
\section*{Questões mais difíceis}
\textbf{22) }
Um circuito Schmitt Trigger simples tem a seguinte cara.
\begin{figure}[!h]
\begin{center}
\includegraphics[scale=1]{sch}
\caption{Schimitt que será utilizado}
\end{center}
\end{figure}
\\
E suas equações são:
$$V_o-V_p=R_2\times i$$
$$V_p=R_1\times i$$
Explicitando $V_p$ da equação temos
$$V_p=\left( \frac{R_1}{R_1+R_2}  \right)V_o$$
Sendo que em um schimitt trigger $V_o$ assumirá dois valores , que no projeto diz que são $\pm$ 12 V.$R_2=39K$,$V_p=50mV$
$$50\times 10^{-3}=12 \left( \frac{R_1}{39\times 10^{3}+R_2}  \right)$$
$$4.17\times 10^{-3}(R_1+39\times 10^{3})=R_1$$
$$(1-4.17\times 10^{-3})R_1=162.63$$
$$R_1=163.31 \Omega$$
\begin{figure}[!h]
\begin{center}
\includegraphics[scale=0.3]{22}
\caption{Resultado}
\end{center}
\end{figure}
\\
\begin{figure}[!h]
\begin{center}
\includegraphics[scale=0.3]{22b}
\caption{Retificador de meia onda}
\end{center}
\end{figure}
\\
\newpage
\textbf{23) }\\
Uma onda que alterna 10 e tem -6 como pico, assim que entendi. E as tensões variam de -6 V a 4 V.\\\\
Esse é um retificador de meia onda de precisão que corta o ciclo positivo. Quando o ciclo positivo ocorre a saída do amplificador operacional é cortada pelo diodo  e o AmpOp não funciona. Quando vem o ciclo negativo a realimentação funciona e a saída do AmpOp tem 0.7 a mais de tensão. Fazendo com que o sistema entregue o sinal de entrada.
\\
\begin{figure}[!h]
\begin{center}
\includegraphics[scale=0.3]{23}
\caption{Com erros de aproximação e erros praticos}
\end{center}
\end{figure}
\\
\newpage
\textbf{24) }\\
Não sei explicar :)
\begin{figure}[!h]
\begin{center}
\includegraphics[scale=0.3]{24}
\caption{Resposta do circuito }
\end{center}
\end{figure}
\\
\\\\
\textbf{25) 26) 27)}
Regulador de tensão e filtros não caem
\end{document}