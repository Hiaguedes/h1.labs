\documentclass[11pt,a4paper]{article}
\title{Avalia\c{c}\~ao 2 de Mec\^anica Aplicada}
\author{Hiago Riba Guedes RGU:11620104\\ Lucas Priori RGU:11311093}
\date{Data limite : 20/04/2017}
\usepackage{tikz}
\begin{document}
\maketitle

\textbf{Quest\~ao 3.48}\\
Passos:\\
\\
1. Desenhar uma conex\~ao entre os pontos A e B, o que ir\'a formar os pontos $A_1B_1$ , $A_2B_2$ , $A_3B_3$ mostrados abaixo  \\
2. Desenhar linhas entre os pontos $A_1A_2$ e $A_2A_3$\\
3. Achar o ponto m\'edio das retas geradas e tra\c{c}ar uma perpendicular as duas e marcar o ponto de interse\c{c}\~ao entre elas,chamar de $O_1$ \\
4. Repetir o mesmo porcesso da 2 e da 3 para as linhas $B_1B_2$ e $B_2B_3$,chamar o ponto de $O_2$\\
5. Conectar $O_1$ com $A_1$ e $O_2$ com $B_2$\\
6. Note que se formou um poligono $O_1AB0_2$e que tem como valores \\
Conex\~ao terra:$O_1O_2$=20.895mm \\
$AB$=52mm \\
$AO_1$=127.051mm\\
$BO_2$=120.234mm\\
7. Verificar a condi\c{c}\~ao de Grashof \\
127.051 + 20.895 $\leq$ 52 + 120.234\\
O que atende a classe 1 de Grashof \\
8. Selecione um ponto na linha $O_2B$ \\
9. Escolher um ponto a uma distancia consider\'avel de $O_2$ e este ser\'a nosso piv\^o \\
10. Desenhe um circulo com centro em $O_2$ e com final na dire\c{c}\~ao de $O_1$ \\
11. Tra\c{c}ar uma reta entre $C_1$ e $C_3$(ponto formado pelo limite angular do movimento nescess\'ario estabelecido pelo projeto,e formando uma extens\~ao dessa reta pois ela ser\'a nosso eixo motor\\
12. Selecionar um ponto $O_3$ arbitr\'ario nessa reta formada,no caso ele foi escolhido 20 unidades ap\'os o come\c{c}o da base principal\\
13. Desenhar um c\'irculo com centro em $O_3$ com raio igual a metade da dist\^ancia da reta $C_1C_3$ ,ap\'os isso marque os pontos $D_1$ e $D_2$ que s\~ao os extremos desse c\'irculo e $D_1$ e $D_2$ s\~ao os pontos  limites da manivela \\


\begin{tikzpicture}
\draw (0,0) rectangle (8.35,9);%principal
\draw (2.025,9) rectangle (6.325,13.3);%retangulo 1
\draw (2.875,11.15) -- (5.475,11.15);%linha A1B1

\draw (8.1845,9.41) -- (11.225,12.45);
\draw (11.225,12.45) -- (14.265,9.41);%retangulo 2 
\draw (14.265,9.41) -- (11.225,6.369);
\draw (11.225,6.369) -- (8.1845,9.41);
\draw (10.306,10.329) -- (12.144,8.4905);%A2B2

\draw (10.225,1.75) rectangle (14.525,6.05);%retangulo 3
\draw (12.375,2.6) -- (12.375,5.2);%linha A3B3

%linha ligando os A's
\draw[dashed] (2.875,11.15) -- (10.306,10.329);%A1A2
\draw[dashed] (10.306,10.329) -- (12.375,5.2);%A2A3

%linha ligando os B's
\draw[dashed] (5.475,11.15) -- (12.144,8.4905);%B1B2
\draw[dashed] (12.144,8.4905) -- (12.375,2.6);%B2B3

%traçando as perpendiculares
\draw[dashed,blue] (6.5905,10.7395) -- ++(-96.3046:5.7);%perpendicular de A1A2
\draw[dashed,blue] (11.3405,7.9645) -- ++(203.63:6);%perpendicular de A2A3
%----------------------------------------------------------
\draw[dashed,red] (8.8095,9.82025) -- ++(-111.74:5.2);%perpendicular de B1B2
\draw[dashed,red] (12.2595,5.54525) -- ++(182.245:5.2);%perpendicular de B2B3

\draw[fill] (6.019,5.630) circle [radius=0.08];
\draw[fill] (7.023,5.341) circle [radius=0.08];

\draw (2.875,11.15) -- (6.019,5.63);%O1 pra A1
\draw (5.475,11.15) -- (7.023,5.341);%O2 pra B1

\draw (7.023,5.341) -- ++(-45:2.5);%C1
\draw[dashed] (6.019,5.63) -- ++(180:1.5);%C3

\draw[dashed] (8.791,3.573) arc (-45:-186.058:2.5);

\draw (8.791,3.573) -- (4.519,5.63);%25.711°

\draw (4.519,5.63) -- ++(154.289:6);% extensão da reta C1C3

\draw (1.176,7.24) circle [radius=2.371];

%pontos-------------------------------------------
\node[above left] at (2.875,11.15) {$A_1$};
\draw[fill] (2.875,11.15) circle [radius=0.08];
\node[above right] at (10.306,10.329) {$A_2$};
\draw[fill] (10.306,10.329) circle [radius=0.08];
\node[above right] at (12.375,5.2) {$A_3$};
\draw[fill] (12.375,5.2) circle [radius=0.08];

\node[above right] at (5.475,11.15) {$B_1$};
\draw[fill] (5.475,11.15) circle [radius=0.08];
\node[above right] at (12.144,8.4905) {$B_2$};
\draw[fill] (12.144,8.4905) circle [radius=0.08];
\node[above right] at (12.375,2.6) {$B_3$};
\draw[fill] (12.375,2.6) circle [radius=0.08];

\node[above right] at (6.019,5.63) {$O_1$};
\draw[fill] (6.019,5.63) circle [radius=0.08];
\node[above right] at (7.023,5.341) {$O_2$};
\draw[fill] (7.023,5.341) circle [radius=0.08];
\node[above right] at (1.176,7.24) {$O_3$};
\draw[fill] (1.176,7.24) circle [radius=0.08];

\node[above right] at (8.791,3.573) {$C_1$};
\draw[fill] (8.791,3.573) circle [radius=0.08];
\node[above right] at (4.519,5.63) {$C_3$};
\draw[fill] (4.519,5.63) circle [radius=0.08];

\node[above left] at (-0.96,8.2686) {$D_1$};
\draw[fill] (-0.96,8.2686) circle [radius=0.08];
\node[above right] at (3.312,6.2114) {$D_2$};
\draw[fill] (3.312,6.2114) circle [radius=0.08];
%--------------------------------------------------

\end{tikzpicture}
Figura feita com a escala de 1:20\\
\\
Considerando o ponto (0,0) como a base inferior esquerda da base principal temos a lista de pontos achados logo abaixo:\\
$A_1$=(2.875,11.15)........$D_1$=(-0.96,8.2686)\\
$A_2$=(10.306,10.329)......$D_2$=(3.312,6.2114\\
$A_3$=(12.375,5.2)\\
$B_1$=(5.475,11.15)\\
$B_2$=(12.144,8.4905)\\
$B_3$=(12.375,2.6)\\
$O_1$=(6.019,5.63)\\
$O_2$=(7.023,5.341)\\
$O_3$=(1.176,7.24)\\
$C_1$=(8.791,3.573) \\
$C_3$=(4.519,5.63)\\

\end{document}

