%%%%%%%%%%%%%%%%%%%%%%%%%%%%%%%%%%%%%%%%%
% a0poster Portrait Poster
% LaTeX Template
% Version 1.0 (22/06/13)
%
% The a0poster class was created by:
% Gerlinde Kettl and Matthias Weiser (tex@kettl.de)
% 
% This template has been downloaded from:
% http://www.LaTeXTemplates.com
%
% License:
% CC BY-NC-SA 3.0 (http://creativecommons.org/licenses/by-nc-sa/3.0/)
%
%%%%%%%%%%%%%%%%%%%%%%%%%%%%%%%%%%%%%%%%%

%----------------------------------------------------------------------------------------
%	PACKAGES AND OTHER DOCUMENT CONFIGURATIONS
%----------------------------------------------------------------------------------------

\documentclass[a0,portrait]{a0poster}

\usepackage{multicol} % This is so we can have multiple columns of text side-by-side
\columnsep=100pt % This is the amount of white space between the columns in the poster
\columnseprule=3pt % This is the thickness of the black line between the columns in the poster

\usepackage[svgnames]{xcolor} % Specify colors by their 'svgnames', for a full list of all colors available see here: http://www.latextemplates.com/svgnames-colors

\usepackage{times} % Use the times font
%\usepackage{palatino} % Uncomment to use the Palatino font

\usepackage{graphicx} % Required for including images
\graphicspath{{/home/hiago/Documents/}} % Location of the graphics files
\usepackage{booktabs} % Top and bottom rules for table
\usepackage[font=small,labelfont=bf]{caption} % Required for specifying captions to tables and figures
\usepackage{amsfonts, amsmath, amsthm, amssymb} % For math fonts, symbols and environments
\usepackage{wrapfig} % Allows wrapping text around tables and figures
\usepackage[utf8]{inputenc}


\begin{document}

%----------------------------------------------------------------------------------------
%	POSTER HEADER 
%----------------------------------------------------------------------------------------

% The header is divided into two boxes:
% The first is 75% wide and houses the title, subtitle, names, university/organization and contact information
% The second is 25% wide and houses a logo for your university/organization or a photo of you
% The widths of these boxes can be easily edited to accommodate your content as you see fit

\begin{minipage}[b]{0.75\linewidth}
\veryHuge \color{NavyBlue} \textbf{Localização de Robôs Móveis via Filtro de Kalman} \color{Black}\\\\ % Title
%\Huge\textit{An Exploration of Complexity}\\[2cm] % Subtitle
\huge \textbf{Hiago Riba Guedes}\\[0.5cm] % Author(s)
\huge Universidade Católica de Petrópolis\\[0.4cm] % University/organization
\huge {Bolsista PIBIC/CNPq no LNCC/MCTIC}\\
\huge{Número do projeto:800333/2016-0}\\
\Large \texttt{hiagoguedes94@gmail.com}

\end{minipage}
%
\begin{center}
%\begin{minipage}[b]{0.25linewidth}
\includegraphics[scale=0.8]{logobitmap}
\hspace{34cm}
\includegraphics[scale=0.6]{lncc}
%\end{minipage}
\end{center}


\vspace{1cm} % A bit of extra whitespace between the header and poster content

%----------------------------------------------------------------------------------------

\begin{multicols}{1} % This is how many columns your poster will be broken into, a portrait poster is generally split into 2 columns

%----------------------------------------------------------------------------------------
%	ABSTRACT
%----------------------------------------------------------------------------------------
\begin{center}
\color{Navy} % Navy color for the abstract
\LARGE
\textbf{Resumo}
\end{center}

\LARGE O objetivo desse trabalho é utilizar técnicas de estimação de estado  no estudo de localização de robôs móveis para fins de navegação local.\\\\



%----------------------------------------------------------------------------------------
%	INTRODUCTION
%----------------------------------------------------------------------------------------

\color{SaddleBrown} % SaddleBrown color for the introduction
\begin{flushleft}
\textbf{Introdução} 
\end{flushleft}
$$$$
\LARGE
\par O desenvolvimento de novas tecnologias computacionais e algoritmos de estimação e aprendizado tem impulsionado de maneira vertiginosa o campo da robótica móvel. Contribui também para isso, o grande interesse de empresas tal como a Google no desenvolvimento de carros autoguiados, a utilização de robôs em hospitais na ajuda de idosos e  até mesmo em robôs de limpeza de baixo custo. Esse crescente interesse pode ser também constatado pelo surgimento de grandes empresas tal como a SpaceX.\par 

  Esses são alguns dos elementos que têm contribuído no crescimento da pesquisa nessa área. Apesar do desenvolvimento vertiginoso nessa área, existem ainda diversos problemas a serem resolvidos,como por exemplo, no cenário da localização de robôs.\\




%----------------------------------------------------------------------------------------
%	OBJECTIVES
%----------------------------------------------------------------------------------------

\color{DarkSlateGray} % DarkSlateGray color for the rest of the content
\begin{flushleft}

\textbf{Objetivos}
\end{flushleft}
$$$$
\LARGE

Este trabalho tem como objetivo realizar um estudo sobre localização de robôs móveis e como aplicá-la para realizar simulações e testes em situações reais. O ambiente será composto exclusivamente de paredes , obstáculos diversos e terrenos planos (indoor) e o ambiente poderá ser composto por marcações no circuito, onde tais recursos irão auxiliar na confirmação da localização. Porém tal implementação dependerá do andamento do projeto do robô. Então, a princípio, serão utilizados sensores ultrassônicos e o robô deverá ser capaz de reconhecer paredes e obstáculos e agir de acordo com a aferição.\\\\Os objetivos específicos serão:\\
\begin{enumerate}
\item Construir um protótipo de robô com a capacidade nescessária para a aplicação
\item Aplicar filtros para as leituras dos sensores
\item Detectar e reconhecer paredes e obstáculos
\item Realizar movimentação ótima utilizando a leitura dos sensores. 
\end{enumerate}
\LARGE
\par Em princípio , esse trabalho pode ser visto como uma etapa importante na implementação do algoritmo SLAM (Simulation Localization And Mapping),que introduz mais inteligência e autonomia ao robô. A abordagem dessa etapa dependerá do tempo nescessário para conclusão das etapas anteriores.

\begin{center}[!h]\vspace{1cm}
\includegraphics[width=0.85\linewidth]{plano}
\captionof{figure}{\color{Green} Figura explicativa em simulação do modelo esperado para o projeto}
\end{center}\vspace{1cm}

%----------------------------------------------------------------------------------------

\end{multicols}
\end{document}\grid
\grid
