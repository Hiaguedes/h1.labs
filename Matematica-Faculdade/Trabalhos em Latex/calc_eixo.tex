\documentclass[100pt]{article}
\begin{document}

C\'alculo do eixo,chaveta\\

Eixo sujeito a tor\c{c}\~ao , $K_f$ adotado igual a 1,6 como indicado na p\'agina 125 do Carvalho; tamb\'em adotamos o comprimento para o mesmo de 10 cm ,por bom senso e pelo padr\~ao visto em outras m\'aquinas do tipo.\\\\

No caso esse eixo ter\'a uma mudan\c{c}a de se\c{c}\~ao e teremos que levar isso em considera\c{c}\~ao para calcularmos as medidas nescess\'arias do projeto\\
\\
 Por quest\~oes de c\'alculo adotaremos raio de filete para mudan\c{c}a de se\c{c}\~ao de 3,5mm, o que nos d\'a um \'indice de sensibilidade (q) de 0.9 pela figura 19 na p\'agina 123,aplicando na f\'ormula $K_f=1+q(K_t-1)$,achamos $K_t=1,67$. \\
Como temos a seguinte f\'ormula abaixo\\
$$\tau =K_t \frac{M}{Z}$$
Sendo $Z=\frac{\pi D^3}{16}$ e M=1158 kgf.mm\\\\

Substituindo os valores temos que d=22.6mm \\
O que nos d\'a $\frac{r}{d}\approx0.16$ e estipulando $\frac{D}{d}=1,33$ (o que nos daria D=29.66mmm) 
\\\\ 
Para chaveta,utilizaremos uma de se\c{c}\~ao retangular ,para isso observaremos a tabela da p\'agina 275, e como temos um di\^ametro acima de 22 mm n\'os adotaremos ent\~ao uma se\c{c}\~ao de 8x7 [mm] \\
Para o c\'alculo do comprimento da chaveta adotaremos o maior valor oferecido por uma das f\'ormulas abaixo\\
$$L=\frac{T.\sigma_c}{\delta.T_{10}.\sigma_{adm}}$$
$$1,25D \le L \le 2D$$
\\
Para a primeira utilizaremos $\delta =0,9$ pois se trata de um cubo de a\c{c}o e $\sigma_c=1000$ $ kgf/cm^2$, para o $T_{10}$ dado na mesma tabela e utilizaremos o menor valor do intervalo,pois ela nos dar\'a o maior comprimento.Ent\~ao temos
$$L=\frac{1158.1000}{0,9.38.2160}=15,67mm$$
Para a segunda f\'ormula temos :
$$1,25.22,6\le L \le 2.22,6$$
$$28,6mm\le L \le 45,2 mm$$
Fazendo uma m\'edia ar\'itm\'etica entre os valores m\'inimo e m\'aximo temos um L=36,9 mm, por quest\~oes de super dimensionamento adotaremos o segundo valor, esse valor ser\'a importante para procurarmos uma polia e uma engrenagem  com espessura comercial equivalente. \\
--------------------------------------------------------------------------------------------------------\\
Considera\c{c}\~oes faltantes
velocidade de corte do disco=$\frac{\pi.D_{tarugo}.n}{1000}=\frac{\pi.25,4.5100}{1000}=406,96 m/s$\\
Diagrama das for\c{c}as aplicadas(mas s\'o tem momento aplicado)\\
Defini\c{c}\~ao de rolamentos e polias\\
C\'alculo para os elementos de fixa\c{c}\~ao(como parafusos,soldas,acoplamentos) e de mola , se tiver\\
Caso n\~ao contenha mola , especificar no desenho ou na memoria que a mesa \'e feita de ferro fundido e que o peso dos componentes j\'a segura por si s\'o a montegem toda\\
\\
PARA O DESENHO\\
Normas t\'ecnicas para solda\\
Extens\~ao para a capa protetora da serra e das correias\\
Mor\c{c}a horizontal com angula\c{c}\~ao (ou n\~ao)\\
Aproveitar a folha para mostrar todos os componentes projetados,assim como o professor mostrou em sala de aula\\
Continua\c{c}\~ao da mesa com o rasgo indicando onde a serra ir\'a entrar na hora de cortar o tarugo
\end{document}