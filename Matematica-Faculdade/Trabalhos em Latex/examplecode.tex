\documentclass[12pt]{article} % increase font size

\usepackage{listings}          % for creating language style
\input{arduinoLanguage.tex}    % adds the arduino language

%% Define an Arduino style fore use later %%
\lstdefinestyle{myArduino}{
  language=Arduino,
  %% Add other words needing highlighting below %%
  morekeywords=[1]{},                  % dark green
  morekeywords=[2]{FILE_WRITE},        % light blue
  morekeywords=[3]{SD, File},          % bold orange
  morekeywords=[4]{open, exists},      % orange
  %% The lines below add a nifty box around the code %%
  frame=shadowbox,                    
  rulesepcolor=\color{arduinoBlue},
}

\begin{document}

\section{SD Card Example}
Below is a snippet of arduino code.  It was added  via the input of the external
 file arduinoLanguage.tex.  A custom style was then created at the top of this
 file so that the non-built-in functions and classes like SD and open() would highlight
 properly.  The style also adds the pretty box around the code.\\

\begin{lstlisting}[style=myArduino]
#include <SD.h>

File logfile;
byte logPin = 10;

void setup() {
  SD.begin(logPin);

  ///////// Create a new file //////////
  char filename[] = "LOGGER00.CSV";
  for (int i = 0; i < 100; i++) {
    filename[6] = i/10 + '0';    // number the file
    filename[7] = i%10 + '0';    //
    if ( SD.exists(filename)==false ) { // only open a new file if it doesn't exist
      logfile = SD.open(filename, FILE_WRITE); 
      break;  // leave the loop!
    }
  }
}

void loop() {
  // put your main code here, to run repeatedly:
  logfile.println("Hello");
  logfile.flush();
  delay(1000);
}
\end{lstlisting}


\end{document}