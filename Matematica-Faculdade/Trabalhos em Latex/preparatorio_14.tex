\documentclass[11pt,a4paper]{article}
\usepackage[utf8]{inputenc}
\title{Preparatório 14 - Eletrônica}
\author{Hiago Riba Guedes RGU:11620104}
\date{Professor:Guilherme Garcia}
\usepackage{graphicx}
\usepackage{circuitikz}
\usepackage{pgfplots}  
\usetikzlibrary{shapes,arrows}
\usepackage{subfigure}
\graphicspath{{/home/hiago/Desktop/}}
\begin{document}
\maketitle
\begin{center}
$ {\begin{array}{cc}
   V_{BE}=0.7V &V_{CE_{max}}=30V    \\
   P_{C_{max}}=16W &I_{C_{max}}=3A   \\
   V_{Z}=10V &P_{Z}=1 W   \\
   V=15V\pm 10 \%     \\
  \end{array} } $
  \end{center} 
\begin{circuitikz} \draw
(0,0) node[npn,rotate=90] (npn) {}
(npn.base) node[anchor=east] {}
(npn.collector) node[anchor=south] {}
(npn.emitter) node[anchor=north] {};
\draw(npn.emitter)to[short](2,0)to[R=$R_e$](2,-2)node[sground]{}to[short,-o](3,-2);
\draw(2,0)to[short,-o](3,0);
\draw(npn.base)to [short,*-](0,-1);
\draw(0,-2)node[sground]{} to[zD](npn.base)to[short](-1,-0.85)to [R=$R_s$](-3,-0.85)to[short](-3,0);
\draw (-5,-2)to[american voltage source,l=15V$\pm 10\%$](-5,0) to [short](npn.collector);
\draw(-5,-2)to[short,-*](0,-2);
%\draw(npn.base)to[v=v](npn.emitter);
\end{circuitikz}
\\\\
 Pelo desenho $V_{CB}$=$V_{R_s}$=5 V\\
 Logo pelas tensões serem iguais , a corrente será igual
 \\\\
	Como $V_Z$=10 e $V_{BE}$=0.7 , $V_{saida}$ é a subtração dos dois logo é 9.3 V. Uma vez que entre a base e o emissor de um transistor existe um diodo que faz cair a tensão em 0.7 V.
	\\Como a fonte pode variar de 13.5 V a 16.5 V e a potência no diodo não pode passar de 1 W , calcularemos $R_s$ tomando como medida o valor máximo da fonte.\\\\
	1 W=10$\times I_z$\\
	$I_{zmax}$=0.1A\\
	16.5-10=$I_{zmax}\times R_s$
	\\\\
	Logo $R_{smin}=65 \Omega$\\
	Quanto maior o Resistor menos se sobrecarrega o diodo.\\
	Calculando ganho do transistor\\
	\\
	A corrente do emissor deverá ser de até 500mA.\\
	$I_E=(\beta+1)\times I_B$\\
    $I_B$ é controlado por $R_s$\\
	Com o nosso $R_s$ mínimo de 65 $\Omega$,teríamos uma corrente de base maior que 500 mA,e como $I_E=I_B+I_C$ e a saída é justamente $I_E$.Então não poderemos usar esse resistor e sim um maior.Pra limitar a base e podemos atender a corrente de saída especificada.\\\\
	Usando-se um resistor de 500$\Omega$\\
	Temos $I_B$=0.01 A\\
	Jogando na fórmula , o ganho($\beta$) do transistor deverá ser de 49 para atender ao projeto.\\
	Cada $R_s$ resultará em um transistor com ganhos diferentes. Deverá ser mais seguro optar por transistores com ganhos maiores , logo resistores maiores deverão ser colocados.\\
	Com o ganho do transistor calculado podemos calcular $R_e$ que fará com que a tensão fique próxima aos 9V que queremos.\\\\
	$I_e=I_B\times(\beta + 1)$\\
	$I_e=0.01\times(50)$\\
	$I_e=0.5 A $\\
	9=0.5$\times R_e$\\
	$R_e$=18$\Omega$
	
\end{document}