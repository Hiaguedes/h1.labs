\documentclass[11pt,a4paper]{article}
\usepackage[utf8]{inputenc}
\title{Preparatório 3 - Eletrônica}
\author{Hiago Riba Guedes RGU:11620104}
\date{Professor:Guilherme Garcia}
\usepackage{graphicx}
\usepackage{circuitikz}
\usepackage{pgfplots}  
\usetikzlibrary{shapes,arrows}
\usepackage{subfigure}
\graphicspath{{/home/hiago/Desktop/}}

\begin{document}
\maketitle
\textbf{4.1-}\\\\
\begin{circuitikz}
\draw(0,0) node[sground]{} ;
\draw(0,0) to[battery=$V_1$] (0,3) to[R=$R_s$](3,3);
\draw(3,0) to[zD*,l=$V_z$](3,3) (3,0) node[sground]{};
\draw(3,3)to [short,-o](4.5,3);
\draw(3,0)to [short,-o](4.5,0);
\end{circuitikz}
\\\\\\
\textbf{4.2-}\\\\
Materiais Dados:\\
Diodo Zener BZX79C9V1-$V_z$=9 V,P=0.5W\\
Resistor 220$\Omega$-2W\\
Potenciômetro decada\\
Fonte DC\\
Voltímetro\\
\\
\begin{circuitikz}
\draw(0,0) node[sground]{} ;
\draw(0,0) to[battery=$V_1$] (0,3) to[R=$R_s$](3,3);
\draw(3,0) to[zD*,l=$V_z$](3,3) (3,0) node[sground]{};
\draw(3,3)to [short](4.5,3);
\draw(3,0)to [short](4.5,0);
\draw(4.5,0) to[R=$R_c$] (4.5,3);
\end{circuitikz}
\\
$R_s=220 \Omega$\\
$I_c=\frac{9}{R_c}$\\
$V_{R_s}=V_f-9$\\
$I_s=\frac{Vf-9}{220}=I_t$\\
$P_{R_s}=2W$;P=RI$^2$;2=220.$I^2_s$\\\\
9,0909.10$^{-3}$=$\frac{(V_f-9)^2}{220^2}$\\
440=$V_f^2-18V_f+81$\\
$V_f^2-18V_f-359=0$\\
$\Delta =1760$\\
9$\pm \frac{\sqrt{\Delta}}{2}$\\
$V_{f1}=29.976 V$ e $V_{f2}=-11.976 V$
\\\\
$V_{f_{max}}$=29.976 V\\
$I_{z_{max}}=\frac{0.5}{9}=55.55mA$\\
$I_z=I_t-I_c$\\\\
$55.55mA=\frac{29.976-9}{220}$-$\frac{9}{R_c}$\\
$R_c=226.156 \Omega$ ; Resistor comercial próximo=$220 \Omega$\\
Range :De $1 \Omega$ a $220\Omega $
\\\\\\
\textbf{4.3-}\\\\
\begin{circuitikz}
\draw(0,0) node[sground]{} ;
\draw(0,0) to[battery=$14V$] (0,3) to[R=$R_s$](3,3);
\draw(3,0) to[zD*,l=$V_z$](3,3) (3,0) node[sground]{};
\end{circuitikz}
\\Onde $V_z$=9 V.
\\$I_z$=5 mA.
\\
\\
$V_{fonte}-V_z=R_s.I_z$\\
14-9=$R_s$.5mA\\
$R_s$=1000$\Omega$
\\\\
\textbf{4.4-}\\\\
$P_z$=500mV\\
$V_f$=20V\\\\
20-9=1000.$I_z$\\
$I_z$=11mA\\
P=9.11mA=99mW\\
Sim,é suficiente.
\end{document}