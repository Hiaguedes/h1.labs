\documentclass[11pt,a4paper]{article}
\usepackage[utf8]{inputenc}
\title{Preparatório 3 - Eletrônica II \\Comparadores de tensão }
\author{Hiago Riba Guedes \\RGU:11620104 \\ Turma:E-ELE-A07}
\date{Professor:Paulo Leite}
\usepackage{graphicx}
\usepackage{circuitikz}
\usepackage{pgfplots}  
\usetikzlibrary{shapes,arrows}
\usepackage{subfigure}
\usepackage{hyperref}
\usepackage[margin=15mm]{geometry}
\graphicspath{{/home/hiago/Desktop/}}
\begin{document}
\maketitle
\begin{center}
\begin{circuitikz} 
       \draw
  (0, 0) node[op amp] (opamp) {}
  (opamp.-) to[R] (-3, 0.5)to [short] ++(-1,0) node[ left]{$V_{in}$}
  (opamp.+) to[short] ++(-5,0) to [R=$P1_A$]++(0,2) to[R=$10 K\Omega$]++(0,2) node[above]{+15V} 
  (opamp.+) to[short,-*] ++(-5,0)node[left]{$V_{ref}$} to [R=$P1_B$]++(0,-2) to[R=$10 K\Omega$]++(0,-2)to++(0,-0.5) node[below]{-15V} 
  (opamp.out) node[right] {$V_o$}
  ;
\end{circuitikz}
\end{center}

O trabalho é calcular o valor de $P1_A$ e $P1_B$ com o valor de tensão de comparação $V_{ref}$ de 2.5 V sabendo que a $P1_A + P1_B = 10K\Omega $ \\\\
Fazendo lei dos nós temos:\\\\
$$15-V_{ref}=(10\times 10^3 + P1_A)\times i_a$$
$$V_{ref}-(-15)=(10\times 10^3 + P1_B)\times i_b$$
\\\\
Já que queremos que o comparador tenha uma referência de 2.5V substituimos então $V_{ref}$ com esse valor, ficando então:\\
$$12.5=(10\times 10^3 + P1_A)\times i_a$$
$$17.5=(10\times 10^3 + P1_B)\times i_b$$
Uma vez que pode se considerar que a impedância de entrada das entradas de um amplificador operacional tende ao infinito , podemos considerar que $i_a=i_b$, na prática o amplificador retém aproximadamente 1$\mu$A de $i_a$ mas para efeito de cálculo essa corrente é desprezível então dividindo as equações acima ficamos com:\\
$$\frac{12.5}{17.5}=\frac{10\times 10^3 + P1_A}{10\times 10^3 + P1_B}$$
Pode-se resolver então de duas formas:\\
Uma é , simplesmente igular os numeradores e os denominadores se assegurando que eles estão na mesma potência sem perder a proporcionalidade.\\
$$12.5 \times 10^3=10\times 10^3 + P1_A \rightarrow P1_A=2.5K\Omega$$
$$17.5 \times 10^3=10\times 10^3 + P1_B \rightarrow P1_B=7.5K\Omega$$
O que corrobora a relação entre $P1_A$ e $P1_B$\\\\

Mas como uma boa prática usaremos a relação entre os mesmos e faremos a conta normalmente.\\
Então ficamos com:\\
$$0.7143=\frac{10\times 10^3 + P1_A}{10\times 10^3 + 10\times 10^3 -P1_A}$$
$$0.7143 \times( 20\times 10^3 -P1_A)=10\times 10^3 + P1_A $$
$$14286-10000= 1.7143P1_A $$
$$P1_A=2.5K\Omega$$
Logo:\\
$$P1_B=7.5K\Omega$$
Então o circuito projetado fica com o seguinte esquemático:\\
\begin{center}
\begin{circuitikz} 
       \draw
  (0, 0) node[op amp] (opamp) {}
  (opamp.-) to[R] (-3, 0.5)to [short] ++(-1,0) node[ left]{$V_{in}$}
  (opamp.+) to[short] ++(-5,0) to [R=$2.5K\Omega$]++(0,2) to[R=$10 K\Omega$]++(0,2) node[above]{+15V} 
  (opamp.+) to[short] ++(-5,0) to [R=$7.5K\Omega$]++(0,-2) to[R=$10 K\Omega$]++(0,-2)to++(0,-0.5) node[below]{-15V} 
  (opamp.out) node[right] {$V_o$}
  ;
\end{circuitikz}
\end{center}

Logo se $V_{in}$ for maior que a referência a saída tende ao valor zero ,uma vez que o valor de ganho do operacional tende a -$\infty$, ou valor booleano 0.Considerando que -$V_{ss}$ está no terra. Ou que o AmpOp é específico para um comparador proprio para circuitos digitais.\\
E se a referência for maior que a $V_{in}$ o valor tende a 5 V ,uma vez que o valor de ganho do operacional tende a +$\infty$ , ou valor booleano 1.Considerando que +$V_{ss}$ está em 5V. Ou que o AmpOp é específico para um comparador proprio para circuitos digitais.
\end{document}