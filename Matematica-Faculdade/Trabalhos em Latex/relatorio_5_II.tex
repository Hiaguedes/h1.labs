\documentclass[11pt]{article}

\usepackage[utf8]{inputenc} % Required for inputting international characters
\usepackage[T1]{fontenc} % Output font encoding for international characters
\usepackage{circuitikz}
\usepackage[portuguese]{babel}
%\usepackage[margin=10mm]{geometry}
\usepackage{mathpazo} % Palatino font
\usepackage{graphicx}
\usepackage{pgfplots}
\usepackage[margin=20mm]{geometry}
\graphicspath{{/home/hiago/Desktop/Eletronica_2/images/}}
\begin{document}

%----------------------------------------------------------------------------------------
%	TITLE PAGE
%----------------------------------------------------------------------------------------

\begin{titlepage} % Suppresses displaying the page number on the title page and the subsequent page counts as page 1
	\newcommand{\HRule}{\rule{\linewidth}{0.5mm}} % Defines a new command for horizontal lines, change thickness here
	
	\center % Centre everything on the page
	
	%------------------------------------------------
	%	Headings
	%------------------------------------------------
	
	\textsc{\LARGE Universidade Católica de Petrópolis}\\[1.5cm] % Main heading such as the name of your university/college
	
	\textsc{\Large Centro de Engenharia e Computação}\\[0.5cm] % Major heading such as course name
	
	\textsc{\large Relatório da quinta experiência}\\[0.5cm] % Minor heading such as course title
	
	%------------------------------------------------
	%	Title
	%------------------------------------------------
	
	\HRule\\[0.4cm]
	
	{\huge\bfseries  Circuitos Integradores}\\[0.4cm] % Title of your document
	
	\HRule\\[1.5cm]
	
	%------------------------------------------------
	%	Author(s)
	%------------------------------------------------
	
	\begin{minipage}{0.4\textwidth}
		\begin{flushleft}
			\large
			\textit{Aluno}\\
			Hiago Riba Guedes % Your name
		\end{flushleft}
	\end{minipage}
	~
	\begin{minipage}{0.4\textwidth}
		\begin{flushright}
			\large
			\textit{RGU}\\
			11620104 % Supervisor's name
		\end{flushright}
	\end{minipage}
	
	% If you don't want a supervisor, uncomment the two lines below and comment the code above
	%{\large\textit{Author}}\\
	%John \textsc{Smith} % Your name
	
	%------------------------------------------------
	%	Date
	%------------------------------------------------
	
	\vfill\vfill\vfill % Position the date 3/4 down the remaining page
	Petrópolis\\
	{\large\today} % Date, change the \today to a set date if you want to be precise
	
	%------------------------------------------------
	%	Logo
	%------------------------------------------------
	
	%\vfill\vfill
	%\includegraphics[width=0.2\textwidth]{placeholder.jpg}\\[1cm] % Include a department/university logo - this will require the graphicx package
	 
	%----------------------------------------------------------------------------------------
	
	\vfill % Push the date up 1/4 of the remaining page
	
\end{titlepage}
\newpage

%\tableofcontents
%\newpage

\section{Resumo}
O presente trabalho faz parte do 5º relatório presente na ementa composta pela disciplina de Laboratório de Eletrônica II ministrada pelo professor  Paulo Cesar Lopes Leite no dia 3 de Abril de 2018 
para a turma E-ELE-A07 na instituição 
Universidade Católica de Petrópolis. Seu objetivo é de estudar o comportamento dos circuitos integradores de tensão que é mais uma das aplicações com amplificadores operacionais.
\section{Teoria}
Circuitos Integradores são circuitos que como o nome diz  integram a sua tensão de entrada.São circuitos importantes para controle de motores. Uma vez que ele é um dos estágios de um PID. E sua função de conseguir guardar seu passado é de extrema importância para esse tipo de aplicação.
\begin{figure}[!h]
\begin{center}
\begin{circuitikz} 
       \draw
  (0, 0) node[op amp] (opamp) {}
  (opamp.-)to[short]++(-1,0)to[short]++(0,1.5)to[short]++(1,0)to[C=$C_1$]++(2.4,0) to(opamp.out)
  (opamp.out)to[short]++(2,0)node[right]{$V_o$}
  (opamp.-)to[short]++(-2,0)to[R=$R_1$]++(-1,0)to[short]++(-1,0)node[left]{$V_i$}
  (opamp.+)to[short]++(0,-.5) node[sground]{}
  ;  
  \end{circuitikz}
  \end{center}
  \caption{Exemplo de circuito integrador}
  \end{figure}
  \\
 Para entender como esse circuito consegue integrar ,primeiro temos que relembrar de Circuitos 1 qual a corrente que passa em um capacitor.
 Sabemos que a definição de corrente é dada por 
 $$	I_c=\frac{\Delta q}{\Delta t}$$
 Isolando $\Delta q$
 $$\Delta q=I_c\Delta t$$
 E a definição de um capacitor é dada por
 $$C=\frac{\Delta q}{\Delta V}$$
 Isolando $\Delta q$
 $$\Delta q=C\Delta V$$
 Igualando as duas equações chegamos que a corrente que passa em um capacitor é dada por:
 $$I_c=C\frac{\Delta V}{\Delta t}$$
 Para intervalos de tempo infinitesimais chegamos :
 $$I_c=C\frac{dV}{dt}$$
 Agora analisando nosso circuito integrador temos que enxergar a seguinte relação.
 $$v_+=v_-=0$$
 Aplicando a lei dos nós temos.
 $$V_i-0=R_1\times i$$
 E analisando a corrente que passa no capacitor $C_1$ vemos que a corrente que passa no mesmo é negativa, uma veaz que a diferença de tensão na mesma é 0-$V_0$.Então podemos substituir essa corrente na equação acima
 $$V_i=-R_1C_1\frac{dV_o}{dt}$$
 $$-\frac{1}{R_1C_1}\int V_i=V_o$$
 Que nos dá a equação da saída de um amplificador integrador.\\
 Na prática colocamos um resistor em paralelo com o capacitor, pois como podemos aplicar uma tensão constante ou de frequência muito baixa ,pode acontecer de o capacitor reter muita carga e extrapolar sua tensão nominal,podendo até explodir.Então entra-se um resistor que limita essa frequência e faz nosso capacitor ser mais seguro e operacional.
 \section{Experimento }
 Pediu-se então para realizar a montagem em protobard do circuito abaixo.\\

 E pediu-se logo após para que vissemos o que acontece quando variássemos sua frequência e analisássemos seu comportamento.Pois se a frequência for muito alta em uma onda quadrada a área entre os ciclos tenderá a ser tão pequena que mal dará para  o integrador entrar em ação. Sendo formada na saída um sinal praticamente uma constante zero. Já se aplicarmos um sinal de frequência muito baixa , a área entre os sinais será tão grande a ponto de chegar em um ponto que a saída mostrará sinais saturados em sua saída.Não sendo mais um integrador.Então pedia-se na experiência seus limites.\\
 \section{Resultados}
 Os resultados encontrados foram:\\\\\\
 \begin{tabular}{|c|c|}
\hline 
 Frequência mínima & 25.7 Hz\\
 \hline
 Frequência máxima & 7650 KHz\\
 \hline
\end{tabular}  
\end{document}