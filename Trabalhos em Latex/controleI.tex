\documentclass[11pt]{article}

\usepackage[utf8]{inputenc} % Required for inputting international characters
\usepackage[T1]{fontenc} % Output font encoding for international characters
\usepackage{circuitikz}
\usepackage[portuguese]{babel}
%\usepackage[margin=10mm]{geometry}
\usepackage{mathpazo} % Palatino font
\usepackage{graphicx}
\usepackage{pgfplots}
\usepackage[margin=20mm]{geometry}
\graphicspath{{/home/hiago/Desktop/Eletronica_2/images/}}
\begin{document}

%----------------------------------------------------------------------------------------
%	TITLE PAGE
%----------------------------------------------------------------------------------------

\begin{titlepage} % Suppresses displaying the page number on the title page and the subsequent page counts as page 1
	\newcommand{\HRule}{\rule{\linewidth}{0.5mm}} % Defines a new command for horizontal lines, change thickness here
	
	\center % Centre everything on the page
	
	%------------------------------------------------
	%	Headings
	%------------------------------------------------
	
	\textsc{\LARGE Universidade Católica de Petrópolis}\\[1.5cm] % Main heading such as the name of your university/college
	
	\textsc{\Large Centro de Engenharia e Computação}\\[0.5cm] % Major heading such as course name
	

	
	%------------------------------------------------
	%	Title
	%------------------------------------------------
	
	\HRule\\[0.4cm]
	
	{\huge\bfseries  Trabalho de Controle e Servomecanismos I}\\[0.4cm] % Title of your document
	
	\HRule\\[1.5cm]
	
	%------------------------------------------------
	%	Author(s)
	%------------------------------------------------
	
	\begin{minipage}{0.4\textwidth}
		\begin{flushleft}
			\large
			\textit{Aluno}\\
			Hiago Riba Guedes \\% Your name
			Erico Gonçalves 
		\end{flushleft}
	\end{minipage}
	~
	\begin{minipage}{0.4\textwidth}
		\begin{flushright}
			\large
			\textit{RGU}\\
			11620104 \\
			11410598
		\end{flushright}
	\end{minipage}
	
	% If you don't want a supervisor, uncomment the two lines below and comment the code above
	%{\large\textit{Author}}\\
	%John \textsc{Smith} % Your name
	
	%------------------------------------------------
	%	Date
	%------------------------------------------------
	
	\vfill\vfill\vfill % Position the date 3/4 down the remaining page
	Petrópolis\\
	{\large\today} % Date, change the \today to a set date if you want to be precise
	
	%------------------------------------------------
	%	Logo
	%------------------------------------------------
	
	%\vfill\vfill
	%\includegraphics[width=0.2\textwidth]{placeholder.jpg}\\[1cm] % Include a department/university logo - this will require the graphicx package
	 
	%----------------------------------------------------------------------------------------
	
	\vfill % Push the date up 1/4 of the remaining page
	
\end{titlepage}
\newpage

\tableofcontents
\newpage
\section{Introdução}
Este trabalho tem como objetivo aplicar conceitos vistos durante as aulas de Controle e Servomecanismos I ministradas pelo professor Giovanne Quadrelli na Universidade Católica de Petrópolis. A aplicação escolhida para esse trabalho (ainda a escolher) para a análise do controle do nosso sistema foi utilizado o Matlab.

\section{Teoria}
A disciplina de Controle e Servomecanismos é primordial para a obtenção da autonomia de qualquer sistema complexo,sendo essa uma das cadeiras mais importantes para quem deseja cursar Engenharia de Controle e Automação, uma vez que dominados os conceitos dessa disciplina pode-se avançar para a construção de um robô, automatizar processos de uma residência, automatizar um sistema de irrigação de solos,etc... 

A príncipio um sistema de controle é um sistema onde se aplica uma entrada que pode ser uma entrada de tensão, uma vazão de água, um valor de temperatura e com isso deseja-se obter uma saída de preferência em outra grandeza física. Como um carro ,onde aplicamos uma pressão de entrada pelo acelerador e temos uma velocidade angular em sua saída . Porém um carro convencional como ainda é o comum hoje em dia ainda está distante de ser considerado um sistema autônomo. Então mesmo que se tenha um tacômetro , um carro a grosso modo pode ser considerado um sistema de malha aberta,pois os sensores utilizados dentro do mesmo não são utilizados para o aperfeiçoamento do mesmo. 

Um sistema de malha fechada consiste em utilizar-se dessa medida afim de comparar-la com um setpoint (que pode ser estipulada tanto por um operador quanto pelo próprio sistema embarcado), com o objetivo de fazer com que um motor ,aproveitando o exemplo do carro, consiga alcançar uma velocidade constante. Pois o sensoriamento consegue captar uma série de amostras durante o percursso , um sistema embarcado calcula o erro entre o setpoint desejado e o valor real e de posse desse erro o sistema manda o quanto de sinal é nescessário para corrigir esse erro. Esse sistema embarcado pode ser tanto um microcontrolador PIC ,como um CLP ou até mesmo um microprocessador.

As formas de se obter um controlador para um determinado sistema é relativamente simples, uma vez que as técnicas utilizadas são a muito tempo utilizadas pelas industrias, porém seu amplo conhecimento acaba sendo restrito na maioria das vezes ao meio acadêmico pois nele se envolve conhecimento de sinais, de elétrica , programação e mecânica. Sendo nescessário ao engenheiro sua confecção tanto de hardware quanto de software para a obtenção de um bom controlador. Para isso o engenheiro precisa de um bom modelo dinâmico que vai reger o comportamento do sistema a ser controlado a fim de se estudar a dita função de transferência do sistema e é a partir dessa função que conseguimos saber se o sistema está eficaz ou não , caso não seja utiliza-se o controlador para fazer com que ele seja.

Atualmente, existem controladores que se utilizam de técnicas de Inteligência Artificial como Machine Learning. Porém o objetivo do trabalho é de abordar os conhecimentos vistos durante o curso ,esse trabalho se manterá apenas em abordar a teoria clássica do PID e como ela funciona no sistema em que se desejará controlar. Uma vez que controladores do tipo PID e do tipo PI ainda são utilizados em grande escala na industria.

\subsection{Controlador PID}
Os controladores PID ou também proporcionais integral derivativo são controladores como seu nome diz que se utilizam das propriedades dessas funções afim de melhorar a resposta transitória do sistema,e sua aplicação pode ser feita tanto por um software quanto por um hardware;utilizando-se capacitores e indutores ou amplificadores operacionais (mais usual).
$$u(t)=K_p e(t)+K_i \int e(t)dt +K_d \frac{de(t)}{dt}$$
Aplicando Laplace 

$$U(s)=K_p E(s) +\frac{K_iE(s)}{s}+sK_dE(s)$$  

Onde E(s) é o sistema a ser controlado no domínio da frequência,tornando a sua análise nesse domínio mais fácil de ser feita .

Basicamente ,os três tipos de parâmetros servem para corrigir o estado transitório e ainda colocar-lo no ponto correto de operação. De maneira mais analítica os parâmetros contribuem da seguinte maneira.\\\\
 \begin{center}
\begin{tabular}{|c|c|c|}
\hline
 & Em excesso & Em falta  \\
\hline
Proporcional &	Menos Overshoot;Mais Lento;Mais Instável & Mais Overshoot;Mais Rápido;Menos Instável	\\
\hline
Integrativo & Mais Overshoot;Mais Instável; Mais rápido & Menos Overshoot;Menos Instável;Mais Lento	\\
\hline
Derivativo & Mais Overshoot;Mais Rápido	& Menos Overshoot;Mais Lento	  \\
\hline 
\end{tabular}
\end{center}

Pela tabela fica mais fácil de entender, mais o controle proporcional ele nunca estabilizará o sistema exceto em frequências muito altas o que quebraria a máquina. Pois basicamente o que o controle proporcional faz é: O sistema percebe que o erro encontra-se acima do setpoint e com isso manda abaixar o sinal ,há um ponto onde o erro fica abaixo do esperado fazendo-o subir e assim vai sucessivamente. 

O Integrativo ele já corrige esses erros de maneira mais inteligente, pois o parâmetro proporcional ele é obrigatório no sistema ,trazendo consigo essa oscilação característica. O Integrativo consegue fazer um histórico desses erros acumulativos e consegue zerar a variação do erro sendo essa parcela importantíssima para a obtenção da precisão.

O derivativo serve para corrigir alguns problemas causados pelo parâmetro integrativo ,como a lentidão da resposta ,pois ele age como uma certa previsão de como o erro vai se comportar em alguns passos adiante ,normalmente essa parcela tende a ser bem menor que as outras pois essa certa "mágica de previsão" como ela age proporcional a frequência ela tende a ser bem amplificada em seu valor final, causando mais problemas do que ajudando. Por isso cabe aqui a observação de um estudo de projeto para a obtenção dos parâmetros corretos. 


Em muito dos casos o parâmetro integrador ou o derivativo são suprimidos da equação isso se deve bastante a aplicação onde o Controlador será empregado. Em motores pequenos como em um hobby motor (utilizado para um time de futebol de robôs móveis não-holonômicos) o termo derivativo é frequentemente tirado do mesmo,pois a resposta do mesmo é bastante rápida naturalmente e sua implementação pode gerar overshootings desnecessários ao sistema. O controle PD é mais utilizado quando precisa-se de uma resposta rápida do sistema e a precisão não é de suma importância.
\newpage
\section{Procedimento Experimental}
\end{document}