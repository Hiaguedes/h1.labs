\documentclass[11pt,a4paper]{article}
\title{Kalman para tracking}
\author{Hiago Riba Guedes }
\date{Data limite :2018}
\usepackage{tikz}
\usepackage[utf8]{inputenc}
\usepackage{gensymb}
\usepackage{pgfplots}
\begin{document}
\maketitle
\begin{center}
\textbf{Esboço para o trabalho}
\end{center}
1-Estudo teórico sobre filtro de Kalman e controle discreto\\
2-Uso de simulação computacional (provavelmente em C ou em processing[caso faça um mini projeto de um robo]) \\
3-Aplicação em projeto (com arduino)(projeto do robo?)\\
3.1-projeto de fonte, comunicação,(shield bluetooth?) ,diagrama elétrico, sensores usados,saída de resultados(?),mapeamento??? \\
4-Considerações práticas\\
\end{document}