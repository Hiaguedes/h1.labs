\documentclass[11pt,a4paper]{article}

\usepackage[svgnames]{xcolor} % Specify colors by their 'svgnames', for a full list of all colors available see here: http://www.latextemplates.com/svgnames-colors

\usepackage{times} % Use the times font
%\usepackage{palatino} % Uncomment to use the Palatino font

\usepackage[font=small,labelfont=bf]{caption} % Required for specifying captions to tables and figures
\usepackage{amsfonts, amsmath, amsthm, amssymb} % For math fonts, symbols and environments
\usepackage[utf8]{inputenc}
\usepackage{color}
\begin{document}

50 HP=37285 W\\
W=F$\times $ v\\
W=T $\times \omega $

Duas engrenagens e a saída é de 400 rpm\\
Fazendo relacão para achar a rotacão na entrada temos:\\\\
$v_{pinhao}=v_{coroa}$\\
$\omega_p r_p =\omega_c r_c$\\
$400 \times r_p= \omega_c 2r_p$\\
$\omega_c=200$\\\\
Encontrando entao a rotacao da coroa encontramos\\
$\omega_c=200 \frac{2\pi}{60}=20.944 rad/s$\\
Então o torque na coroa é de 
$T=\frac{37285}{20.94}=1806.35 J$\\
Torquen no pinhão é 903.176 J\\\\
Com esses torques sendo aplicados nas engrenagens iremos calcular seus parâmetros para números de dentes variados, afim de ver se a engrenagem resiste ao torque ao qual será solicitado...\\\\

Fazendo então um par de engrenagem de 12 e 24 dentes , temos

\end{document}