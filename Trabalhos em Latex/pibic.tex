\documentclass[100pt]{article}


\title{Avalia\c{c}\~ao PIBIC/LNCC}
\author{Hiago Riba Guedes}
\begin{document}
\maketitle
\emph{Quest\~ao 1} \\
1.   Aplicando a regra da cadeia temos: $$tan'(x).sin'(tan(x))$$ 
E por sua vez $$tan'(x)=(\frac{sin(x)}{cos(x)})'=\frac{(sin'(x).cos(x) - cos'(x).sin(x))}{cos^2(x)}=\frac{cos^2(x) + sin^2(x)}{cos^2(x)}=\frac{1}{cos^2(x)} $$
O que nos da: \\
$$\frac{1}{cos^2(x)}.cos(tan(x))= \frac{cos(tan(x))}{cos^2(x)} $$ \\
2. Sendo$$j=\sqrt{-1}$$
 $$ \sqrt{1 + j}=(\sqrt{2}\angle{45})^{1/2}=\sqrt[4]{2}\left[cos\left(\frac{\frac{\pi}{4} +2k\pi}{2}\right) +jsin\left(\frac{\frac{\pi}{4} +2k\pi}{2}\right)\right]$$
 Para k=0 e 1

k=0
\\
 $$\sqrt[4]{2}\left[cos\left(\frac{\pi}{8}\right) +jsin\left(\frac{\pi}{8}\right)\right]$$
 k=1
 $$\sqrt[4]{2}\left[cos\left(\frac{9\pi}{8}\right) +jsin\left(\frac{9\pi}{8}\right)\right]$$ \\
 
3.  
$$\int e^{-x} = -e^{-x} + c$$
Como o intervalo eh $$[0,\infty)$$
$$-e^{-\infty}-(-e^{-0})=0+1=1 $$
\\
4. Usando a Regra de Integraçao por partes $$\int u.dv=u.v-\int v.du $$ 
E definindo: \\
$$u=x  \textbf{ e } dv=e^{-x}$$
Logo: $$du=1 \textbf{ e } v=-e^{-x} $$
$$ x.-e^{-x}-\int-e^{-x}.1 + c $$
$$-x.e^{-x}+\int e^{-x}+c$$
$$-x.e^{-x}-e^{-x}+c$$
No intervalo de o até infinito temos:
$$-\infty.e^{-\infty}-e^{-\infty} - [-0.e^{-0}-e^{-0}]$$
O que nos da um caso de indeterminacão de 
$$\frac{\infty}{\infty}$$
5. $$\int sin(x)=-cos(x) + c$$
$$\int_0^{\pi} sin(x)=-cos(\pi)+cos(0)=2$$
6. Usando a formula do arco metade$$cos^2(x)=\frac{1}{2}+ \frac{cos(2x)}{2} $$
$$\int_{-\pi/2}^{\pi/2} \frac{1}{2}+\int_{-\pi/2}^{\pi/2} \frac{cos(2x)}{2} $$
Aplicando a regra da substituicao temos $$2x=u \textbf{ e } 2dx=du;dx=\frac{du}{2}$$
$$\int cos(u)du=\frac{1}{2} sin(u)dx=\frac{1}{2}sin(2x)$$
Substituindo:
$$\frac{1}{2}(\frac{\pi}{2} - \frac{-\pi}{2})+\frac{1}{4}sin(2\frac{\pi}{2}-2\frac{-\pi}{2})$$
$$\frac{\pi}{2}+\frac{sin(2\pi)}{4}=\frac{\pi}{2}$$
\\
7. Usando o mesmo raciocinio da 4 ,temos que:
$$dv=x^{2k};u=sin(x);v=\frac{x^{2k+1}}{2k+1};du=cos(x)$$
$$\int x^{2k}sin(x)dx=sin(x).\frac{x^{2k+1}}{2k+1}-\int\frac{x^{2k+1}}{2k+1}.cos(x)$$
$$\int x^{2k}sin(x)dx=sin(x).\frac{x^{2k+1}}{2k+1}-\frac{1}{2k+1}\int x^{2k+1}.cos(x)$$
\\
$$u=x^{2k+1};dv=cos(x);du=(2k+1)x^{2k};v=sin(x)$$\\


\emph{Quest\~ao 2}\\
$$\sum_{k=0}^{n} s^{k}=1+s+s^{2}+s^{3}+s^{4}+...$$
Esta é uma série geométrica com a=1 e r=x e como |x|<1 , isto é uma série convergente .Logo:
 $$\sum_{k=0}^{n} s^{k}=\frac{1}{1-s}$$
 Entao 
 $$\lim_{n \to \infty} f(x)=(1-s)\frac{1}{1-s}=1$$\\
 \emph{Quest\~ao 3}\\
 A funcao f(x)=f(-x) representa matematicamente qualquer funcao par.
E uma propriedade que temos para esse tipo de funcao eh 
$$\int_{-a}^{a} f(x)=2\int_0^{a} f(x)$$
Isto e a area de uma lado da funcao e igual a do outro lado da mesma funcao.
Isto e que 
$$\int_{-a}^0 f(x) = \int_0^{a} f(x)$$


Para a questao temos que 
$$\int_t^{\infty} f(x)=\int_0^{\infty}f(x)-\int_0^t f(x) \textbf{ e } \int_{-\infty}^t f(x)=\int_{-\infty}^0 f(x) +\int_0^t f(x) $$
Somando as duas funcoes temos que :
$$\int_t^{\infty} f(x) + \int_{-\infty}^0f(x)+\int_0^t f(x)=\int_{-\infty}^t f(x) + \int_0^{\infty}f(x)-\int_0^t f(x)$$
So que :
$$\int_{-\infty}^0f(x)=\int_0^{-\infty}f(x)$$
Ficando:
$$\int_t^{\infty} f(x) + \int_0^t f(x)=\int_{-\infty}^t f(x) -\int_0^t f(x)$$
So que :
$$\left|\int_t^{\infty} f(x)\right|>>>\left|\int_0^t f(x)\right|$$
e vice versa\\
Resultando em :
$$\int_t^{\infty} f(x) =\int_{-\infty}^t f(x) $$
\\
\emph{Quest\~ao 4}
\\
$$A=\left[\begin{array}{ccrll}
a_{11}& a_{12} & \ldots &a_{1n}\\
0  & a_{22}& \dots & a_{2n}\\
\vdots &\ddots &\ddots &\vdots\\
0 &\cdots &\cdots& a_{nn}
\end{array}\right]$$
Para se achar os autovalores dessa matriz voce tem que fazer 
$$det(A-\lambda I)=0$$
Uma das propriedades dos determinantes diz que o determinante de uma matriz triangular é igual ao
 produto dos elementos da 
diagonal principal. 
Entao temos que :
$$(a_{11}- \lambda_1)(a_{22}- \lambda_2)(a_{33}- \lambda_3)\cdots(a_{nn}- \lambda_n)=0$$
O que da como resultado de lambda o proprio coeficiente :
Isto e
$$ \lambda_1 =a_{11};\lambda_2 =a_{22};\cdots;\lambda_n =a_{nn};$$
\\
\emph{Quest\~ao 5}
\\
$$X=\left[\begin{array}{ccrll}
a_{11} & \ldots &a_{1n}\\
\vdots &\ddots &\vdots\\
a_{1n} &\cdots & a_{nn}
\end{array}\right]$$
Pede-se para encontrar a inversa de :
$$\left[\begin{array}{cccrll}
1 & \cdots & 0 & a_{1 n+1} & \cdots & a_{1 2n}\\
\vdots & \ddots & \vdots &\vdots & \ddots &\vdots\\
0 & \cdots & 1 & a_{n n+1} & \cdots & a_{n 2n}\\
0 & \cdots & 0 & 1 & \cdots & 0\\
\vdots & \ddots & \vdots &\vdots& \ddots &\vdots\\
0 & \cdots & 0 & 0 & \cdots & 1\\
\end{array}\right]$$
Uma matriz e a sua inversa se multiplicadas ,tem que dar a matriz identidade
$$A.A^{-1}=I$$
$$\left[\begin{array}{ccrll}
I & X\\
0 & I\\
\end{array}\right]
\left[\begin{array}{ccrll}
D & E\\
F & G\\
\end{array}\right]
=
\left[\begin{array}{ccrll}
I & 0\\
0 & I\\
\end{array}\right]
$$
Fazendo as multiplicacoes linha por coluna temos:\\

$$D+XF=I$$
$$E+XG=0$$
$$F=0$$ 
$$G=I 
$$
Com isso temos D=I e E=-A \\
Logo a inversa de A eh;
$$\left[\begin{array}{ccrll}
I & -X\\
0 & I\\
\end{array}\right]$$
\end{document}