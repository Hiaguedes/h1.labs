\documentclass[11pt,a4paper]{article}
\usepackage[utf8]{inputenc}
\title{Preparatório 8 - Eletrônica}
\author{Hiago Riba Guedes RGU:11620104}
\date{Professor:Guilherme Garcia}
\usepackage{graphicx}
\usepackage{circuitikz}
\usepackage{pgfplots}  
\usetikzlibrary{shapes,arrows}
\usepackage{subfigure}
\graphicspath{{/home/hiago/Desktop/}}

\begin{document}
\maketitle
\begin{circuitikz} \draw
(10,10) node[npn] (npn) {}
(npn.base) node[anchor=south] {B}
(npn.collector) node[anchor=east] {C}
(npn.emitter) node[anchor=east] {E};
\draw(9.5,10)to [short,i<=350uA](7,10)to [R=$R_B$](7,14.5)to[short](10,14.5);
\draw(10,10.7)to [short,i<=35mA](10,12)to [R=$220\Omega$](10,15) node[vcc]{$V_{CC}$};
\draw(10,9.5) to[short](10,8)node[sground]{};
\end{circuitikz}
\\
$V_{CE}=5V$\\
$V_{BE}=0.7V$
\\\\
\textbf{4.1-}\\\\
$V_{CE}=V_{CC}-	I_CR_C$\\
5=$V_{CC}$-35.$10^{-3}.220$\\
$V_{CC}=12.7V$\\\\
\textbf{4.2-}\\\\
$V_{CC}=R_B.I_B+V_{BE}$\\
12.7=$R_B$.350.$10^{-6}$+0.7\\
$R_B$=34,286 K$\Omega$
\\\\
\textbf{4.3-}
\\\\
P=R$I^2$\\\\
$P_C=R_C.I_C^2$\\
$P_C=220.(35.10^{-3})^2$\\
$P_C=0.27 W$  Atende\\\\
Calculo análogo para $P_B$\\
$P_B$=34,286K$\Omega$.(350.$10^{-6})^2$\\
$P_B$=0.004W  Atende\\\\
\textbf{4.4-}\\\\
$P_{diss}=V_{CE}.I_C$\\
P=5.$35.10^{-3}$\\
P=0.175W      \\\\É compatível
\end{document}