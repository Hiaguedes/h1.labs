\documentclass[11pt,a4paper]{article}
\usepackage[utf8]{inputenc}
\title{Preparatório 2 - Eletrônica}
\author{Hiago Riba Guedes RGU:11620104}
\date{Professor:Guilherme Garcia}
\usepackage{graphicx}
\usepackage{subfigure}
\graphicspath{{/home/hiago/Desktop/}}

\begin{document}
\maketitle
\textbf{4.1}\\\\
Dopando-se um diodo,pode-se controlar o nível de tensão de ruptura ,então dependendo do nível onde for colocado ,o diodo passa então a conduzir em polarização reversa.\\\\
\begin{figure}[!htb]
\textbf{4.2}\\
\includegraphics[scale=0.8]{regulador_zener}
\end{figure}

\begin{figure}[!htb]
\textbf{4.3}\\\\
$P_max$=400mW\\
$V_z$=9.1 V\\
Fonte max=20V\\
\\
Tensão no resistor=20-9.1=10.9V\\
P=I.U\\
400mW=10.9 x I\\
I=36.7mA\\\\
U=R.I\\
10.9=R.36.7mA\\
$R=297\Omega$\\\\
Comercial mais próximo ; 300$\Omega$
\end{figure}

\begin{figure}[!htb]
\textbf{4.4}\\\\
P=$\frac{1}{4}$W =250 mW\\
250mW=i x 10.9=22.935mA\\


\end{figure}



\end{document}